% !TEX root=week-9-model-ensembles.tex

\documentclass[10pt]{beamer}
\usepackage{../../../../latex/packages/tslwn-preamble}
\usepackage{../../../../latex/packages/tslwn-slides}

\setbeamertemplate{frametitle continuation}{}

\AtEveryBibitem{%
  \clearfield{url}%
}

\title{Ensembles}
\author{Tim Lawson}

\begin{document}
\maketitle

\begin{frame}{Bias and variance}
  A classification error may occur when:
  \\~\
  \begin{itemize}
    \item Feature vectors/per-class distributions \emph{overlap}
    \item The model has \emph{high bias} (is not expressive enough)
    \item The model has \emph{high variance}
          \\
          ~\
  \end{itemize}
  \parencite[][p. 338–9]{Flach2012}
\end{frame}

\begin{frame}{Ensembles}
  \begin{itemize}
    \item Learn multiple models from different versions of the data
    \item[]
          \begin{itemize}
            \item Resample the instances and/or features
            \item Reweight the instances
          \end{itemize}
    \item Aggregate the models' predictions
    \item[]
          \begin{itemize}
            \item Average the scores or probabilities
            \item Choose the majority prediction
          \end{itemize}
    \item Or, learn multiple types of model
  \end{itemize}
\end{frame}

\begin{frame}{Bagging methods}
  A bagging method learns multiple instances of a model from random subsets of
  the data and aggregates the instances' predictions.
  \\~\
  \begin{itemize}
    \item \emph{Pasting}: random subsets of the instances are sampled without replacement
          \parencite{Breiman1999}
    \item \emph{Bagging}: random subsets of the instances are sampled with replacement
          \parencite{Breiman1996}
    \item \emph{Random subspaces}: random subsets of the features are sampled
          \\ \parencite{Ho1998}
    \item \emph{Random patches}: random subsets of the instances and features
          are sampled \parencite{Louppe2012}
          \\~\
  \end{itemize}
  \parencite[][sec. 1.11.3]{zotero-3054}
\end{frame}

\begin{frame}{Boosting methods}
  A boosting method learns multiple instances of a model from weighted versions
  of the data and aggregates the instances' predictions.
  \\~\
  \begin{itemize}
    \item \emph{AdaBoost}: weights are updated based on the error of the
          previous model \parencite{Freund1997}
    \item \emph{Gradient-boosted trees}: weights are updated based on the
          gradient of the loss function, e.g., LightGBM \parencite{Ke2017},
          XGBoost \parencite{Chen2016} \\~\
  \end{itemize}
  \parencite[][sec. 1.11.1]{zotero-3054}
  \\~\

  See also, e.g., \emph{arcing} (``adaptively resample and combine'') and random
  forests \parencite{Breiman1998,Breiman2001}.
\end{frame}

\begin{frame}{Boosting methods}{Code}
  \lstinputlisting[
    language=Python,
    linerange={43-47,48-55}
  ]{src/examples/binary_classifier_ensemble_todo.py}
\end{frame}

\begin{frame}{Multiple types of model}
  The predictions of different types of models can be aggregated by:
  \\~\
  \begin{itemize}
    \item \emph{Voting} or \emph{averaging} (soft voting)
    \item \emph{Stacked generalization} (stacking): using the predictions as
          the feature of a meta-model
    \item \emph{Meta-learning}: learn a model that predicts whether a model
          will perform well on a given task and data
          \\~\
  \end{itemize}
\end{frame}

\begin{frame}[allowframebreaks]{References}
  \renewcommand*{\bibfont}{\footnotesize}
  \printbibliography
\end{frame}

\begin{frame}{Boosting}{Definition}
  \begin{itemize}
    \item $\{x_i \mid i \in 1 .. n\},\ \vec{x}_i \in \mathbb{R}^{n_x}$ are
          instances
    \item $\vec{y}_i \in \{0,1\}^{n_y}$ is a label (one-hot vector)
    \item $f^{(t)} : \mathbb{R}^{n_x} \to \{0,1\}^{n_y}$ is a model
    \item $\vec{\hat{y}}_i^{(t)} = f^{(t)}(\vec{x}_i) \in \{0,1\}^{n_y}$ is a
          prediction (one-hot vector)
    \item $w_i^{(t)} \in \mathbb{R},\ w_i^{(0)} = \frac{1}{n},\
            \sum_{i = 1}^{n}w^{(t)}_i = 1$ is an instance weight
    \item $\epsilon^{(t)} = \sum_{i : \vec{\hat{y}}_i^{(t)} \neq \vec{y}_i}
            w^{(t)}_i \in \mathbb{R}$ is the weighted error of model $f^{(t)}$
    \item $\alpha^{(t)} = f_\alpha(\epsilon^{(t)}) \in \mathbb{R}$ is the weight
          of model $f^{(t)}$
    \item $w^{(t+1)}_i = f_w(w^{(t)}_i,\ \vec{y}_i,\ \vec{\hat{y}}_i^{(t)},\
            \epsilon^{(t)})$ is the updated instance weight
    \item $\vec{\hat{y}}_i = \sum_{t = 1}^{T}\alpha^{(t)}f^{(t)}(x_i)
            \in \{0,1\}^{n_y}$ is the ensemble model prediction
  \end{itemize}
\end{frame}

\begin{frame}{Boosting}{Derivation}
  \begin{itemize}
    \item What should the weights of the models
          $f_\alpha$/\lstinline[language=Python]{get_model_weights} be?
    \item What should the weight updates
          $f_w$/\lstinline[language=Python]{update_weight} be?
          \\~\
  \end{itemize}
  Assume that the weight updates $f_w$ are:
  \begin{equation*}
    w^{(t+1)}_i
    =
    \frac{w^{(t)}_i}{Z^{(t)}}
    \times
    \begin{cases}
      e^{-\alpha^{(t)}} & \text{if } \vec{\hat{y}}_i^{(t)} = \vec{y}_i
      \\
      e^{\alpha^{(t)}}  & \text{otherwise}
    \end{cases}
  \end{equation*}
  This can be simplified with:
  \begin{equation*}
    \delta(\vec{y}_i,\,\vec{\hat{y}}_i^{(t)})
    =
    \begin{cases}
      1  & \text{if } \vec{\hat{y}}_i^{(t)} = \vec{y}_i
      \\
      -1 & \text{otherwise}
    \end{cases}
  \end{equation*}
\end{frame}

\begin{frame}{Boosting}{Derivation}
  The weight updates are:
  \begin{equation*}
    w^{(t+1)}_i
    =
    w^{(t)}_i
    \frac{
    \exp(-\alpha^{(t)}\delta(\vec{y}_i,\,\vec{\hat{y}}_i^{(t)}))
    }{
    Z^{(t)}
    }
  \end{equation*}
  Each update is multiplicative:
  \begin{equation*}
    w^{(T)}_i
    =
    w^{(0)}_i
    \prod_{t=1}^{T}
    \frac{
    \exp(-\alpha^{(t)}\delta(\vec{y}_i,\,\vec{\hat{y}}_i^{(t)}))
    }{
    Z^{(t)}
    }
    =
    \frac{1}{n}
    \frac{
    \exp(-\delta(\vec{y}_i,\,\vec{\hat{y}}_i^{(t)}))
    }{
    \prod_{t=1}^{T}Z^{(t)}
    }
  \end{equation*}
  Each set of instance weights sums to 1:
  \begin{equation*}
    1
    =
    \sum_{i = 1}^{n}w^{(t)}_i
    =
    \sum_{i = 1}^{n}
    \frac{1}{n}
    \frac{
    \exp(-\delta(\vec{y}_i,\,\vec{\hat{y}}_i^{(t)}))
    }{
    \prod_{t=1}^{T}Z^{(t)}
    }
  \end{equation*}
  \begin{equation*}
    \prod_{t=1}^{T} Z^{(t)}
    =
    \frac{1}{n}
    \sum_{i = 1}^{n}
    \exp(-\delta(\vec{y}_i,\,\vec{\hat{y}}_i^{(t)}))
  \end{equation*}
\end{frame}

\begin{frame}{Boosting}{Derivation}
  $\exp(-\delta(\vec{y}_i,\,\vec{\hat{y}}_i^{(t)})) \geq 1$ if $x_i$ is
  misclassified by the ensemble, so $\prod_{t=1}^{T}
    Z^{(t)}$ is an upper bound on the ensemble error.
  \\~\

  $\prod_{t=1}^{T} Z^{(t)}$ could be minimized by minimizing the model error
  $(n)$$Z^{(t)}$:
    \begin{equation*}
      n Z^{(t)}
      =
      \sum_{i = 1}^{n}
      w^{(t)}_i
      \exp(-\alpha^{(t)}\delta(\vec{y}_i,\,\vec{\hat{y}}_i^{(t)}))
    \end{equation*}
    By the definitions of $\epsilon^{(t)}$ and
  $\delta(\vec{y}_i,\,\vec{\hat{y}}_i^{(t)})$:
  \begin{equation*}
    n Z^{(t)}
    =
    \epsilon^{(t)} \exp(\alpha^{(t)})
    +
    (1 - \epsilon^{(t)}) \exp(-\alpha^{(t)})
  \end{equation*}
\end{frame}

\begin{frame}{Boosting}{Derivation}
  Therefore, $Z^{(t)}$ is minimized when:
  \begin{equation*}
    \frac{\partial Z^{(t)}}{\partial \alpha^{(t)}}
    =
    \epsilon^{(t)} \exp(\alpha^{(t)})
    -
    (1 - \epsilon^{(t)}) \exp(-\alpha^{(t)})
    =
    0
  \end{equation*}
  \begin{equation*}
    \exp(2\alpha^{(t)})
    =
    \frac{1 - \epsilon^{(t)}}{\epsilon^{(t)}}
  \end{equation*}
  That is:
  \begin{equation*}
    \alpha^{(t)}
    =
    \frac{1}{2}\ln\left(\frac{1 - \epsilon^{(t)}}{\epsilon^{(t)}}\right)
    ,\
    Z^{(t)}
    =
    2\sqrt{\epsilon^{(t)}(1 - \epsilon^{(t)})}
  \end{equation*}
\end{frame}

\end{document}
