\section{Probabilistic models}
\label{sec:9:probabilistic-models}

Recall that:
\begin{itemize}
  \item $P(Y \mid X)$ is the posterior probability distribution of $Y$ given $X$;
  \item $P(Y, X)$ is the joint probability distribution of $Y$ and $X$;
  \item $P(X \mid Y)$ is the likelihood function; and
  \item $P(X)$ is the prior distribution of $X$.
\end{itemize}

\paragraph{Discriminative and generative models}

\begin{itemize}
  \item A \textit{discriminative} model describes the posterior distribution of
        the target given the input.
        It does not describe the prior distribution of the input.
  \item A \textit{generative} model describes a joint distribution of the target
        and input.
        If the prior distribution of the target can be estimated, then it can
        be described by a likelihood function.
        It can be used to generate data by sampling from the joint distribution.
\end{itemize}

Generative models can do more than discriminative models.
However, joint distributions are harder to learn than conditional distributions
like the posterior distribution because they are described by more probability
values (parameters).
This may be handled by simplifying assumptions like independence but they are
not always appropriate.

\paragraph{Uncertainty}

A probabilistic view treats learning as a procedure that reduces uncertainty.

\subsection{Normal distributions}

\paragraph{Univariate}

\begin{dfn}
  [Univariate normal distribution]
  \label{dfn:9:univariate-normal-distribution}
  \begin{align}
    P(x \mid \mu, \sigma)
     & = \frac{1}{\sqrt{2 \pi} \sigma} \exp \left( - \frac{(x - \mu)^2}{2 \sigma^2} \right) \\
     & = \frac{1}{E} \exp \left( - \frac{z^2}{2} \right) ,\
    E = \sqrt{2 \pi} \sigma
  \end{align}
  \begin{itemize}
    \item $\mu \in \mathbb{R}$ is the mean;
    \item $\sigma \in \mathbb{R}$ is the standard deviation; and
    \item $z = \frac{x - \mu}{\sigma}$ is the $z$-score.
  \end{itemize}
\end{dfn}
If $\mu = 0$ and $\sigma = 1$, then it is the \textit{standard} univariate
normal distribution:
\begin{equation}
  P(x \mid \mu = 0, \sigma = 1)
  = \frac{1}{\sqrt{2 \pi}} \exp \left( - \frac{x^2}{2} \right)
\end{equation}

\paragraph{Multivariate}

\begin{dfn}
  [Multivariate normal distribution]
  \label{dfn:9:multivariate-normal-distribution}
  \begin{equation}
    \begin{split}
      P(\vec{x} \mid \vec{\mu}, \matr{\Sigma})
      = \frac{1}{E_n} \exp \left(
      - \frac{1}{2} (\vec{x} - \vec{\mu})^T \matr{\Sigma}^{-1} (\vec{x} - \vec{\mu})
      \right) ,\\
      E_n = (2 \pi)^{n/2} \sqrt{\det \matr{\Sigma}}
    \end{split}
  \end{equation}
  \begin{itemize}
    \item $\vec{\mu} \in \mathbb{R}^n$ is the mean; and
    \item $\matr{\Sigma} \in \mathbb{R}^{n \times n}$ is the covariance matrix.
  \end{itemize}
\end{dfn}
If $\vec{\mu} = \vec{0}$ and $\matr{\Sigma} = \matr{I}$, then it is the
\textit{standard} multivariate normal distribution:
\begin{equation}
  P(\vec{x} \mid \vec{\mu} = \vec{0}, \matr{\Sigma} = \matr{I})
  = \frac{1}{(2 \pi)^{n / 2}} \exp \left( - \frac{\vec{x} \cdot \vec{x}}{2} \right)
\end{equation}

\subsubsection{Gaussian mixture models}

A Gaussian mixture model is a mixture of $k$ Gaussian distributions.
For $k = 2$, i.e., binary classification,
$X = \{ x_i \mid i = 1 .. n \} = X_+ \cup X_-$.

\paragraph{Univariate}

In the univariate case, the likelihood ratio is:
\begin{equation*}
  \frac{P(X_+)}{P(X_-)}
  = \frac{\sigma_-}{\sigma_+} \exp \left(
  - \frac{1}{2} \left(
    \left( \frac{x - \mu_+}{\sigma_+} \right)^2
    - \left( \frac{x - \mu_-}{\sigma_-} \right)^2
    \right)
  \right)
\end{equation*}

If $\sigma_+ = \sigma_- = \sigma$, then the likelihood ratio is:
\begin{equation*}
  \exp (\gamma (x - \mu)) ,\
  \gamma = \frac{\mu_+ - \mu_-}{\sigma^2} ,\
  \mu = \frac{\mu_+ + \mu_-}{2}
\end{equation*}
and the maximum-likelihood decision threshold, i.e., the value of $x$ such
that the likelihood ratio is 1, is $\mu$.

If the standard deviations of the two Gaussian distributions are different, then
there are two decision boundaries and a non-contiguous decision region for one
of the classes.

\paragraph{Multivariate}

In the multivariate case, the likelihood ratio is:
\begin{equation*}
  \sqrt{\frac{\det\Sigma_+}{\det\Sigma_-}}
  \exp \left(-\frac{1}{2} \left(
    \left(\vec{x} - \vec{\mu}_+ \right)^T \Sigma_+^{-1}
    \left(\vec{x} - \vec{\mu}_+ \right) -
    \left(\vec{x} - \vec{\mu}_- \right)^T \Sigma_-^{-1}
    \left(\vec{x} - \vec{\mu}_- \right)
    \right) \right)
\end{equation*}

If $\matr{\Sigma}_+ = \matr{\Sigma}_- = \matr{I}$, i.e., for each class, the
features are uncorrelated and have unit variance, then the likelihood ratio is:
\begin{equation*}
  \exp \left(
  -\frac{1}{2} \left(
    \norm{\vec{x} - \vec{\mu}_+}^2 - \norm{\vec{x} - \vec{\mu}_-}^2
    \right)\right)
\end{equation*}
and the maximum-likehood decision boundary, i.e., the values of $\vec{x}$ such
that the likelihood ratio is 1, is the hyperplane equidistant from $\vec{\mu}_+$
and $\vec{\mu}_-$.

This is the same as the decision boundary for the basic linear classifier.
In other words, for uncorrelated Gaussian features with unit variance, the basic
linear classifier is \textit{Bayes-optimal}.

\paragraph{Distances and probabilities}

The normal distribution demonstrates the connection between the geometric and
probabilistic views of models.

The multivariate normal distribution (definition~\ref{dfn:9:multivariate-normal-distribution})
translates the Mahalanobis distance (definition~\ref{dfn:8:mahalanobis-distance})
into a probability:

\begin{equation}
  P(\vec{x} \mid \vec{\mu}, \matr{\Sigma})
  = \frac{1}{E_n} \exp\left( - \frac{1}{2} D_M (\vec{x}, \vec{\mu} \mid \matr{\Sigma})^2 \right)
\end{equation}

For $n = 2$ dimensions, the standard normal distribution translates the
Euclidean distance (definition~\ref{dfn:8:euclidean-distance}) into a
probability:

\begin{equation}
  P(\vec{x} \mid \vec{0}, \matr{I})
  = \frac{1}{E_2} \exp\left( - \frac{1}{2} D_2 (\vec{x}, \vec{0})^2 \right)
\end{equation}

The logarithm transforms a multiplicative scale into an additive scale.
Thus, the negative logarithm of the Gaussian likelihood can be interpreted as a
squared distance:

\begin{equation}
  - \ln P(\vec{x} \mid \vec{\mu}, \matr{\Sigma})
  = \ln E_d + \frac{1}{2} D_M (\vec{x}, \vec{\mu} \mid \matr{\Sigma})^2
\end{equation}

\subsection{TODO}

\begin{itemize}
  \item Linear regression
  \item Categorical variables
  \item Categorical probability distributions
  \item Decision rules
  \item Naïve Bayes
  \item Logistic regression
\end{itemize}