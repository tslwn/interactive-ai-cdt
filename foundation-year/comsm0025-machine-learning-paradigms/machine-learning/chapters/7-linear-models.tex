\section{Linear models}

\begin{itemize}
      \item Linear models are defined in terms of the geometry of the instance space.
      \item Real-valued features are not generally intrinsically geometric.
      \item However, we can use geometric concepts to structure the instance space
            (e.g., lines and planes) and represent similarity by distance.
\end{itemize}

Linear models are simple:

\begin{itemize}
      \item They are parametric: they have a fixed structure that is defined by
            numeric parameters that are learned from the training data.
            By contrast, tree and rule models are non-parametric: their structure
            is not fixed prior to learning.
      \item They are stable (have low variance): small variations in the training
            data have a small effect on the learned model.
            Tree models have high variance.
      \item They are unlikely to overfit the training data because they have
            relatively few parameters (have high bias).
            However, they sometimes underfit the training data.
\end{itemize}

\subsection{The least-squares method}

The least-squares method can be used to learn linear models for classification
and regression.
It finds a function estimator that minimises the sum of squared residuals
(differences between the actual and estimated values).

\paragraph{Univariate linear regression}

Let $\{ (x_i, y_i) \mid i \in 1 .. n \}$ be a set of instances.
Approximate the true function $f(x_i) = y_i$ by a linear function
$f^\prime(x_i) = a + b x_i$.
Univariate linear regression finds $a, b$ such that the sum of squared
residuals $\sum_{i = 1}^{n} (y_i - (a + b x_i))^2$ is minimized.

When the sum of squared residuals is minimized, its partial derivatives with
respect to $a$ and $b$ are zero:

\begin{alignat}{2}
      \frac{\partial}{\partial a} \sum_{i = 1}^{n} (y_i - (a + b x_i))^2
       & = - 2 \sum_{i = 1}^{n} (y_i - (a + b x_i))
       & = 0
      \\
      \frac{\partial}{\partial b} \sum_{i = 1}^{n} (y_i - (a + b x_i))^2
       & = - 2 \sum_{i = 1}^{n} (y_i - (a + b x_i)) x_i
       & = 0
\end{alignat}

TODO

\begin{itemize}
      \item Translation does not affect the regression coefficient, only the intercept.
            We can zero-centre the $x$-values by subtracting the mean $\bar{x}$.
      \item If we normalize $x$ to have unit variance, then the regression
            coefficient is the covariance between the normalized $x$ and $y$.
\end{itemize}

The least-squares solution is equivalent to the maximum likelihood estimate
given the assumptions that the true function is linear but normally-distributed
noise is added to the instance $y$-values.
If noise is added to only the $y$-values, then it is called \textit{ordinary}
least squares, which has a unique solution.
If noise is added to both $x$- and $y$-values, then it is called \textit{total}
least squares, which does not necessarily have a unique solution.

Zero-centred matrix, scatter matrix, covariance matrix.

\paragraph{Multivariate linear regression}

Matrix form and homogeneous coordinates.
Transformation to decorrelate, centre and normalize features.
If the features are assumed to be uncorrelated, a multivariate linear regression
problem decomposes into a set of univariate linear regression problems.
It is computationally expensive to invert the scatter (covariance) matrix.

\paragraph{Regularization}

Least-squares regression can be unstable.
Instability demonstrates a tendency to overfit.
Regularization helps to avoid overfitting by constraining the weight vector.

\begin{itemize}
      \item Shrinkage: makes the average magnitude of the weights small.
            This adds a scalar parameter to the diagonal of the scatter matrix,
            which improves the numerical stability of matrix inversion.
            Least-squares regression with shrinkage is called ridge regression.
      \item Lasso (least absolute shrinkage and selection operator): This adds the
            sum of the absolute weights ($L_1$ regularization).
            This makes the magnitude of some weights smaller but sets others to
            zero, i.e., it favours sparse solutions.
\end{itemize}

\subsection{The perceptron}

\subsection{Support vector machines}

\subsection{Obtaining probabilities from linear classifiers}


\subsection{Notes}

When to favour models with different characteristics?
E.g., the quantity and quality training data.

Normalization (zero-centre, unit variance)
Write up the equivalencies between correlation coefficients etc.

Regularization
Relation to, e.g., Bayesian priors
Technically, e.g., sparsity (Occam's razor).
Regularization changes the optimal solution (it's included in the loss)

Correlation (in the extreme case, two copies of the same feature) — your
problem is underspecified, i.e., there are infinitely many solutions (which are
combinations of the two).
`Spikiness of the fitness landscape' (high variance).
Regularization: decreasing dependence on the data, i.e., increasing bias and
decreasing variance.

When to choose ridge (L2) vs lasso (L1) regularization?
An elastic net uses a weighted combination of the two where the weight is a
hyperparameter that you can tune.

Why does lasso produce sparse solutions?
With Euclidean distance, the set of points at equal distance is a circle.
If you change the exponent, e.g., Minkowski at d = 3, that changes shape.
E.g. d = 1 `pulls you' towards a solution on one of the axes, i.e., towards one
or the other feature instead of a combination of both (i.e., sparsity).
Vector field analysis.

LP-norm where p is an integer.

Multivariate.

