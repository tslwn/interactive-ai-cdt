\subsection{Feynman diagrams}

Use perturbation theory to evaluate the following expressions in terms of
integrals over products of factors $iG(\tp, \tpp)$ and draw the corresponding
Feynman diagrams.

\begin{enumerate}[(a)]
  \setcounter{enumi}{1}
  \item $\avg{x(t_1) x(t_2) \exp\left(-\epsilon\ihbar\int_0^t x(\tp)^6 \dtp \right)}$
        neglecting terms of order $\epsilon^2$ and higher
\end{enumerate}

The Taylor expansion of the exponential is:
\begin{align*}
  \exp\left(-\epsilon\ihbar\int_0^t x(\tp)^6 \dtp \right)
   & =
  1 - \epsilon\ihbar\int_0^t x(\tp)^6 \dtp + O(\epsilon^2)
\end{align*}
Hereafter $=\cdots + O(\epsilon^2)$ will be written as $\approx\cdots$.
Hence:
\begin{align*}
  \avg{x(t_1) x(t_2) \exp\left(-\epsilon\ihbar\int_0^t x(\tp)^6 \dtp \right)}
   & \approx
  \avg{x(t_1) x(t_2)} - \epsilon\ihbar\int_0^t \avg{x(t_1) x(t_2) x(\tp)^6} \dtp
\end{align*}
There is one Feynman diagram for $\avg{x(t_1) x(t_2)}$ (figure~\ref{fig:37b1})
and two Feynman diagrams for $\avg{x(t_1) x(t_2) x(\tp)^6}$
(figures~\ref{fig:37b2} and \ref{fig:37b3}).
The diagram in figure~\ref{fig:37b1} has multiplicity $1$.
There are $(6 - 1)\doublefact$ ways to choose pairs of the $6$ $\tp$ vertices;
hence, the diagram in figure~\ref{fig:37b2} has multiplicity $15$.
In the diagram in figure~\ref{fig:37b3}, there are $\binom{6}{1}$ ways to
choose a pair of the $1$ $t_1$ and $6$ $\tp$ vertices, $\binom{5}{1}$ ways to
choose a pair of the $1$ $t_2$ and remaining $5$ $\tp$ vertices, and $(4 -
  1)\doublefact$ ways to choose pairs of the remaining $4$ $\tp$ vertices.
Hence, it has multiplicity $6 \times 5 \times 3 = 90$.

\begin{figure}[ht!]
  \centering
  \begin{tikzpicture}
    \begin{feynman}
      \vertex (t1);
      \vertex [right=1in of t1] (t2);
      \diagram* {
        (t1) -- (t2),
      };
      \node[circle, fill=black, inner sep=1pt] at (t1);
      \node[circle, fill=black, inner sep=1pt] at (t2);
      \node[below] at (t1) {$t_1$};
      \node[below] at (t2) {$t_2$};
    \end{feynman}
  \end{tikzpicture}
  \caption{Feynman diagram for $\avg{x(t_1) x(t_2)}$ of contribution \dots}
  \label{fig:37b1}
\end{figure}

\begin{figure}[ht!]
  \centering
  \begin{tikzpicture}
    \begin{feynman}
      \vertex (t1);
      \vertex [below=1in of t1] (t2);
      \vertex [below right=0.5in and 1in of t1] (tp);
      \diagram* {
      (t1) -- (t2),
      (tp) --[loop, out=000, in=060, min distance=1cm] tp,
      (tp) --[loop, out=120, in=180, min distance=1cm] tp,
      (tp) --[loop, out=240, in=300, min distance=1cm] tp,
      };
      \node[circle, fill=black, inner sep=1pt] at (t1);
      \node[circle, fill=black, inner sep=1pt] at (t2);
      \node[circle, fill=black, inner sep=1pt] at (tp);
      \node[left=0.5cm] at (t1) {$t_1$};
      \node[left=0.5cm] at (t2) {$t_2$};
      \node[below right=0cm and 0.5cm] at (tp) {$\tp$};
    \end{feynman}
  \end{tikzpicture}
  \caption{Feynman diagram for $\avg{x(t_1) x(t_2) x(\tp)^6}$ of contribution \dots}
  \label{fig:37b2}
\end{figure}

\begin{figure}[ht!]
  \centering
  \begin{tikzpicture}
    \begin{feynman}
      \vertex (t1);
      \vertex [below=1in of t1] (t2);
      \vertex [below right=0.5in and 0.5in of t1] (tp);
      \diagram* {
      (t1) -- (tp),
      (t2) -- (tp),
      (tp) --[loop, out=285, in=345, min distance=1cm] tp,
      (tp) --[loop, out=015, in=075, min distance=1cm] tp,
      };
      \node[circle, fill=black, inner sep=1pt] at (t1);
      \node[circle, fill=black, inner sep=1pt] at (t2);
      \node[circle, fill=black, inner sep=1pt] at (tp);
      \node[left=0.5cm] at (t1) {$t_1$};
      \node[left=0.5cm] at (t2) {$t_2$};
      \node[below right=0cm and 0.5cm] at (tp) {$\tp$};
    \end{feynman}
  \end{tikzpicture}
  \caption{Feynman diagram for $\avg{x(t_1) x(t_2) x(\tp)^6}$ of contribution \dots}
  \label{fig:37b3}
\end{figure}

We have that:
\begin{equation}
  \avg{x(\tp) x(\tpp)} = iG(\tp, \tpp)
  \quad\text{where}\quad
  (A^{-1} x)(\tp) = \int_0^t G(\tp, \tpp) x(\tpp) \dtp
\end{equation}
Hence:
\begin{align*}
   & \avg{x(t_1) x(t_2) \exp\left(-\epsilon\ihbar\int_0^t x(\tp)^6 \dtp \right)}
  \\[2ex]
   & \quad \approx
  \avg{x(t_1) x(t_2)} - \epsilon\ihbar\int_0^t \avg{x(t_1) x(t_2) x(\tp)^6} \dtp
  \\[2ex]
   & \quad \approx
  iG(t_1, t_2) - \epsilon\ihbar\int_0^t \left(
  15 iG(t_1, t_2) \left(iG(\tp, \tp)\right)^3 +
  90 iG(t_1, \tp) iG(t_2, \tp) \left(iG(\tp, \tp)\right)^2
  \right) \dtp
  \\[2ex]
   & \quad \approx
  iG(t_1, t_2) - \epsilon\ihbar\int_0^t \left(
  15 G(t_1, t_2) G(\tp, \tp)^3 +
  90 G(t_1, \tp) G(t_2, \tp) G(\tp, \tp)^2
  \right) \dtp
\end{align*}
