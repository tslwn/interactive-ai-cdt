\begin{enumerate}[leftmargin=0cm]
  \setcounter{enumi}{1}
  \item
        \begin{enumerate}
          \item Determine the Feynman diagrams to evaluate
                $I = \avg{x_k^2 e^{- \epsilon \sum_\kp x_\kp^6 }}$
                up to $O(\epsilon)$, their multiplicities, and their contributions to $I$.
        \end{enumerate}
\end{enumerate}

The Taylor expansion of $e^{-\epsilon \sum_\kp x_\kp^6}$ is:
\begin{equation*}
  e^{-\epsilon \sum_\kp x_\kp^6}
  = \sum_{n=0}^{\infty} \frac{1}{n\fact} \left( -\epsilon \sum_\kp x_\kp^6 \right)^n
  = 1 - \epsilon \sum_\kp x_\kp^6 + O(\epsilon^2)
\end{equation*}
Hereafter $=\cdots + O(\epsilon^2)$ will be written as $\approx\cdots$.
Hence $I$ is:
\begin{equation*}
  I \approx \avg{x_k^2 \left(1 - \epsilon \sum_\kp x_\kp^6 \right)} \approx
  \avg{x_k^2} - \epsilon \avg{x_k^2 \sum_\kp x_\kp^6} \approx \avg{x_k^2} -
  \epsilon \sum_\kp \avg{x_k^2 x_\kp^6}
\end{equation*}
We have that, for an average $\avg{x_k^p x_\kp^\pp}$, a Feynman diagram with $m$
connections between $k$ and $\kp$ has multiplicity:
\begin{equation}{}
  \label{eq:2a-multiplicity}
  {p \choose m} {\pp \choose m} m\fact (p - m - 1)\doublefact (\pp - m - 1)\doublefact
\end{equation}
There are two Feynman diagrams for $\avg{x_k^2 x_\kp^6}$ (figures~\ref{fig:2a1}
and \ref{fig:2a2}).
By equation~\ref{eq:2a-multiplicity}, the multiplicities of the two diagrams are
respectively:
\begin{align*}{}
  {2 \choose 0} {6 \choose 0} 0\fact (2 - 0 - 1)\doublefact (6 - 0 - 1)\doublefact & = 15
  \\[2ex]
  {2 \choose 2} {6 \choose 2} 2\fact (2 - 2 - 1)\doublefact (6 - 2 - 1)\doublefact & = 90
\end{align*}
The sum of the multiplicities is $7\doublefact = 105$, as expected.
Hence $I$ is:
\begin{equation*}
  I \approx
  \matr{A}^{-1}_{kk} - \epsilon \left(
  15 \left(\matr{A}^{-1}_{kk}\right) \left(\matr{A}^{-1}_{\kp\kp}\right)^3 +
  90 \left(\matr{A}^{-1}_{k\kp}\right)^2 \left(\matr{A}^{-1}_{\kp\kp}\right)^2
  \right)
\end{equation*}

\begin{figure}
  \centering
  \begin{tikzpicture}
    \begin{feynman}
      \vertex (k);
      \vertex [right=1in of k] (kp);
      \diagram* {
      (k)  --[loop, out=045, in=135, min distance=1.5cm] k,
      (kp) --[loop, out=000, in=060, min distance=1.5cm] kp,
      (kp) --[loop, out=120, in=180, min distance=1.5cm] kp,
      (kp) --[loop, out=240, in=300, min distance=1.5cm] kp,
      };
      \node[circle, fill=black, inner sep=1pt] at (k);
      \node[circle, fill=black, inner sep=1pt] at (kp);
      \node[below] at (k) {$k$};
      \node[below right=0cm and 0.5cm] at (kp) {$\kp$};
    \end{feynman}
  \end{tikzpicture}
  \caption{Feynman diagram for $\avg{x_k^2 x_\kp^6}$ with contribution
    $15 \left(\matr{A}^{-1}_{kk}\right) \left(\matr{A}^{-1}_{\kp\kp}\right)^3$.}
  \label{fig:2a1}
\end{figure}
\begin{figure}
  \centering
  \begin{tikzpicture}
    \begin{feynman}
      \vertex (k);
      \vertex [right=1in of k] (kp);
      \diagram* {
      (k)  --[quarter left] (kp),
      (k)  --[quarter right] (kp),
      (kp) --[loop, out=015, in=075, min distance=1.5cm] kp,
      (kp) --[loop, out=285, in=345, min distance=1.5cm] kp,
      };
      \node[circle, fill=black, inner sep=1pt] at (k);
      \node[circle, fill=black, inner sep=1pt] at (kp);
      \node[left=0cm] at (k) {$k$};
      \node[right=0.5cm] at (kp) {$\kp$};
    \end{feynman}
  \end{tikzpicture}
  \caption{Feynman diagram for $\avg{x_k^2 x_\kp^6}$ with contribution
    $90 \left(\matr{A}^{-1}_{k\kp}\right)^2 \left(\matr{A}^{-1}_{\kp\kp}\right)^2$.}
  \label{fig:2a2}
\end{figure}
