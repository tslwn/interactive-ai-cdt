\begin{enumerate}[leftmargin=0cm]
  \setcounter{enumi}{1}
  \item
        \begin{enumerate}
          \item \label{2a}
                Determine the Feynman diagrams to evaluate $I = \avg{x_k^2 e^{- \epsilon
                        \sum_\kp x_\kp^6 }}$ up to $O(\epsilon)$, their multiplicities, and their
                contributions to $I$.
        \end{enumerate}
\end{enumerate}

The Taylor expansion of $e^{-\epsilon \sum_\kp x_\kp^6}$ is:
\begin{equation*}
  e^{-\epsilon \sum_\kp x_\kp^6}
  = \sum_{n=0}^{\infty} \frac{1}{n\fact} \left( -\epsilon \sum_\kp x_\kp^6 \right)^n
  = 1 - \epsilon \sum_\kp x_\kp^6 + O(\epsilon^2)
\end{equation*}
Hereafter $=\cdots + O(\epsilon^2)$ will be written as $\approx\cdots$.
Hence $I$ is:
\begin{equation*}
  I \approx \avg{x_k^2 \left(1 - \epsilon \sum_\kp x_\kp^6 \right)} \approx
  \avg{x_k^2} - \epsilon \avg{x_k^2 \sum_\kp x_\kp^6} \approx \avg{x_k^2} -
  \epsilon \sum_\kp \avg{x_k^2 x_\kp^6}
\end{equation*}
We have that, for an average $\avg{x_k^p x_\kp^\pp}$, a Feynman diagram with $m$
connections between $p$ $k$ and $\pp$ $\kp$ vertices has multiplicity:
\begin{equation}{}
  \label{eq:2a-multiplicity}
  {p \choose m} {\pp \choose m} m\fact (p - m - 1)\doublefact (\pp - m - 1)\doublefact
\end{equation}
There are two Feynman diagrams for $\avg{x_k^2 x_\kp^6}$ (figures~\ref{fig:2a1}
and \ref{fig:2a2}).
By equation~\ref{eq:2a-multiplicity}, the multiplicities of the two diagrams are
respectively:
\begin{align*}{}
  {2 \choose 0} {6 \choose 0} 0\fact (2 - 0 - 1)\doublefact (6 - 0 - 1)\doublefact & = 15
  \\[2ex]
  {2 \choose 2} {6 \choose 2} 2\fact (2 - 2 - 1)\doublefact (6 - 2 - 1)\doublefact & = 90
\end{align*}
The sum of the multiplicities is $7\doublefact = 105$, as expected.
Hence $I$ is:
\begin{equation*}
  I \approx
  \matr{A}^{-1}_{kk} - \epsilon \left(
  15 \left(\matr{A}^{-1}_{kk}\right) \left(\matr{A}^{-1}_{\kp\kp}\right)^3 +
  90 \left(\matr{A}^{-1}_{k\kp}\right)^2 \left(\matr{A}^{-1}_{\kp\kp}\right)^2
  \right)
\end{equation*}

\begin{figure}[h!]
  \centering
  \begin{tikzpicture}
    \begin{feynman}
      \vertex (k);
      \vertex [right=1in of k] (kp);
      \diagram* {
      (k)  --[loop, out=045, in=135, min distance=1cm] k,
      (kp) --[loop, out=000, in=060, min distance=1cm] kp,
      (kp) --[loop, out=120, in=180, min distance=1cm] kp,
      (kp) --[loop, out=240, in=300, min distance=1cm] kp,
      };
      \node[circle, fill=black, inner sep=1pt] at (k);
      \node[circle, fill=black, inner sep=1pt] at (kp);
      \node[below] at (k) {$k$};
      \node[below right=0cm and 0.5cm] at (kp) {$\kp$};
    \end{feynman}
  \end{tikzpicture}
  \caption{\ref{2a}.
    Feynman diagram for $\avg{x_k^2 x_\kp^6}$ with contribution $15
      \left(\matr{A}^{-1}_{kk}\right) \left(\matr{A}^{-1}_{\kp\kp}\right)^3$.
  }
  \label{fig:2a1}
\end{figure}
\begin{figure}[h!]
  \centering
  \begin{tikzpicture}
    \begin{feynman}
      \vertex (k);
      \vertex [right=1in of k] (kp);
      \diagram* {
      (k)  --[quarter left] (kp),
      (k)  --[quarter right] (kp),
      (kp) --[loop, out=015, in=075, min distance=1cm] kp,
      (kp) --[loop, out=285, in=345, min distance=1cm] kp,
      };
      \node[circle, fill=black, inner sep=1pt] at (k);
      \node[circle, fill=black, inner sep=1pt] at (kp);
      \node[left=0cm] at (k) {$k$};
      \node[right=0.5cm] at (kp) {$\kp$};
    \end{feynman}
  \end{tikzpicture}
  \caption{\ref{2a}.
    Feynman diagram for $\avg{x_k^2 x_\kp^6}$ with contribution $90
      \left(\matr{A}^{-1}_{k\kp}\right)^2 \left(\matr{A}^{-1}_{\kp\kp}\right)^2$.
  }
  \label{fig:2a2}
\end{figure}

\begin{enumerate}[leftmargin=0cm]
  \setcounter{enumi}{1}
  \item
        \begin{enumerate}
          \setcounter{enumii}{1}
          \item \label{2b}
                Given the average $J = \avg{x_k^2 x_\kp^2 e^{- \epsilon \sum_\kpp x_\kpp^6 }}$,
                determine the multiplicities of the Feynman diagrams in figures~\ref{fig:2b1},
                \ref{fig:2b2} and \ref{fig:2b3}.
        \end{enumerate}
\end{enumerate}

For $m = p = \pp$, equation~\ref{eq:2a-multiplicity} reduces to $m\fact$;
hence, the diagram in figure~\ref{fig:2b1} has multiplicity $2\fact = 2$.
There are $(6 - 1)\doublefact = 15$ ways to choose pairs of the $6$ $\kpp$
vertices; hence, the diagram in figure~\ref{fig:2b2} has multiplicity $2 \times
  15 = 30$.
In the diagram in figure~\ref{fig:2b3}, there are ${6 \choose 2}$ ways to
choose pairs of the $2$ $k$ and $6$ $\kpp$ vertices, ${4 \choose 2}$ ways to
choose pairs of the $2$ $\kp$ and remaining $4$ $\kpp$ vertices, and
$2\doublefact$ ways to choose pairs of the remaining $2$ $\kpp$ vertices.
Hence, it has multiplicity ${6 \choose 2} {4 \choose 2} 2\doublefact = 180$.

\begin{figure}[h!]
  \centering
  \begin{tikzpicture}
    \begin{feynman}
      \vertex (k);
      \vertex [right=1in of k] (kp);
      \diagram* {
      (k)  --[quarter left] (kp),
      (k)  --[quarter right] (kp),
      };
      \node[circle, fill=black, inner sep=1pt] at (k);
      \node[circle, fill=black, inner sep=1pt] at (kp);
      \node[left=0cm] at (k) {$k$};
      \node[right=0cm] at (kp) {$\kp$};
    \end{feynman}
  \end{tikzpicture}
  \caption{\ref{2b}.
    Feynman diagram for $\avg{x_k^2 x_\kp^2}$ with multiplicity $2$.
  }
  \label{fig:2b1}
\end{figure}
\begin{figure}[h!]
  \centering
  \begin{tikzpicture}
    \begin{feynman}
      \vertex (k);
      \vertex [right=0.75in of k] (kp);
      \vertex [right=0.75in of kp] (kpp);
      \diagram* {
      (k)  --[quarter left] (kp),
      (k)  --[quarter right] (kp),
      (kpp) --[loop, out=000, in=060, min distance=1cm] kpp,
      (kpp) --[loop, out=120, in=180, min distance=1cm] kpp,
      (kpp) --[loop, out=240, in=300, min distance=1cm] kpp,
      };
      \node[circle, fill=black, inner sep=1pt] at (k);
      \node[circle, fill=black, inner sep=1pt] at (kp);
      \node[circle, fill=black, inner sep=1pt] at (kpp);
      \node[left=0cm] at (k) {$k$};
      \node[right=0cm] at (kp) {$\kp$};
      \node[below right=0cm and 0.5cm] at (kpp) {$\kpp$};
    \end{feynman}
  \end{tikzpicture}
  \caption{\ref{2b}.
    Feynman diagram for $\avg{x_k^2 x_\kp^2 x_\kpp^6}$ with multiplicity $15$.
  }
  \label{fig:2b2}
\end{figure}
\begin{figure}[h!]
  \centering
  \begin{tikzpicture}
    \begin{feynman}
      \vertex (k);
      \vertex [below right=0.25in and 0.25in of k] (kpp);
      \vertex [below left=0.25in and 0.25in of kpp] (kp);
      \diagram* {
      (k)  --[quarter left] (kpp),
      (k)  --[quarter right] (kpp),
      (kp) --[quarter left] (kpp),
      (kp) --[quarter right] (kpp),
      (kpp) --[loop, out=-30, in=030, min distance=1cm] kpp,
      };
      \node[circle, fill=black, inner sep=1pt] at (k);
      \node[circle, fill=black, inner sep=1pt] at (kp);
      \node[circle, fill=black, inner sep=1pt] at (kpp);
      \node[above left] at (k) {$k$};
      \node[below left] at (kp) {$\kp$};
      \node[below right=0.25cm and 0.25cm] at (kpp) {$\kpp$};
    \end{feynman}
  \end{tikzpicture}
  \caption{\ref{2b}.
    Feynman diagram for $\avg{x_k^2 x_\kp^2 x_\kpp^6}$ with multiplicity $90$.
  }
  \label{fig:2b3}
\end{figure}

\begin{enumerate}[leftmargin=0cm]
  \setcounter{enumi}{1}
  \item
        \begin{enumerate}
          \setcounter{enumii}{2}
          \item \label{2c}
                Find a Feynman diagram other than those in figures~\ref{fig:2b1},
                \ref{fig:2b2}, and \ref{fig:2b3} for $\avg{x_k^2 x_\kp^2 x_\kpp^6}$ in which
                the two legs of $\kp$ are not connected to each other.
        \end{enumerate}
\end{enumerate}

See figure~\ref{fig:2c}.

\begin{figure}[h!]
  \centering
  \begin{tikzpicture}
    \begin{feynman}
      \vertex (k);
      \vertex [below right=0.25in and 0.25in of k] (kpp);
      \vertex [below left=0.25in and 0.25in of kpp] (kp);
      \diagram* {
      (k)  -- (kpp),
      (kp) -- (kpp),
      (kpp) --[loop, out=285, in=345, min distance=1cm] kpp,
      (kpp) --[loop, out=015, in=075, min distance=1cm] kpp,
      };
      \node[circle, fill=black, inner sep=1pt] at (k);
      \node[circle, fill=black, inner sep=1pt] at (kp);
      \node[circle, fill=black, inner sep=1pt] at (kpp);
      \node[above left] at (k) {$k$};
      \node[below left] at (kp) {$\kp$};
      \node[right=0.5cm] at (kpp) {$\kpp$};
    \end{feynman}
  \end{tikzpicture}
  \caption{\ref{2c}.
    Feynman diagram for $\avg{x_k^2 x_\kp^2 x_\kpp^6}$.
  }
  \label{fig:2c}
\end{figure}

\begin{enumerate}[leftmargin=0cm]
  \setcounter{enumi}{1}
  \item
        \begin{enumerate}
          \setcounter{enumii}{3}
          \item \label{2d}
                Given the average $\tilde{J} = \avg{x_k^2 x_\kp^2 e^{- \epsilon \sum_\kpp
                        x_\kpp^m }}$, determine the multiplicities of the Feynman diagrams analogous to
                those in figures~\ref{fig:2b1}, \ref{fig:2b2} and \ref{fig:2b3}.
        \end{enumerate}
\end{enumerate}

The diagram in figure~\ref{fig:2b1} does not change and has multiplicity $2$.
There are $(m - 1)\doublefact$ ways to choose pairs of $m$ $\kpp$ vertices;
hence, the diagram analogous to figure~\ref{fig:2b2} has multiplicity $2 (m -
  1)\doublefact$.
In the diagram analogous to figure~\ref{fig:2b3}, there are ${m \choose 2}$
ways to choose pairs of the $2$ $k$ and $m$ $\kpp$ vertices, ${m - 2 \choose
      2}$ ways to choose pairs of the $2$ $\kp$ and remaining $m - 2$ $\kpp$
vertices, and $(m - 4)\doublefact$ ways to choose pairs of the remaining $m -
  4$ $\kpp$ vertices.
Hence, it has multiplicity ${m \choose 2} {m - 2 \choose 2} (m -
  4)\doublefact$.
