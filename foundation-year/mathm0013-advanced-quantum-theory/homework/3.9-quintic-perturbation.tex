\subsection{Quintic perturbation}

\begin{enumerate}[(a)]
  \setcounter{enumi}{3}
  \item Consider a perturbed harmonic oscillator with the Lagrangian:
        \begin{equation}
          L = \half m \dot{x}^2 - \half m \omega^2 x^2 - \epsilon x^n
        \end{equation}
        where $n \geq 3$ is an odd integer.
        Which are the Feynman diagrams for the leading non-vanishing perturbation to
        the propagator, and what are their multiplicities?
\end{enumerate}

The propagator is:
\begin{align*}
  \propone[anh] = \propone[harm] \avg{\exp\left(-\epsilon\ihbar\int_0^t x(\tp)^n \,\dtp \right)}
\end{align*}
The Taylor expansion of the exponential is:
\begin{multline*}
  \exp\left(-\epsilon\ihbar\int_0^t x(\tp)^n \,\dtp \right)
  = 1 - \epsilon\ihbar\int_0^t x(\tp)^n \,\dtp \dots
  \\
  \dots
  - \frac{1}{2}\epsilon^2\frac{1}{\hbar^2}
  \int_0^t\dtp \int_0^t\dtpp
  x(\tp)^n x(\tpp)^n
  + O(\epsilon^3)
\end{multline*}
Hereafter $=\cdots + O(\epsilon^3)$ will be written as $\approx\cdots$.
Hence:
\begin{equation*}
  \avg{\exp\left(-\epsilon\ihbar\int_0^t x(\tp)^n \,\dtp \right)}
  \approx 1
  - \frac{1}{2}\epsilon^2\frac{1}{\hbar^2}
  \int_0^t\dtp \int_0^t\dtpp
  \avg{x(\tp)^n x(\tpp)^n}
\end{equation*}
The term of order $\epsilon$ vanishes because $n$ is odd and hence
$\avg{x(\tp)^n} = 0$.

There are $\half (n + 1)$ Feynman diagrams for $\avg{x(\tp)^n x(\tpp)^n}$.
For the example of $n = 5$, there are $3$ diagrams for $m = \{1, 3, 5\}$
connections between the legs of the $\tp$ and $\tpp$ vertices
(figure~\ref{fig:39a1}).
The multiplicities sum to $(10 - 1)\doublefact = 945$, as expected.
\begin{figure}[ht!]
  \centering

  \begin{tabular}
    {m{6cm}m{5cm}}
    \begin{center}
      Feynman diagram
    \end{center}
     &
    \begin{center}
      Multiplicity
    \end{center}
    \\
    \hline
    \begin{center}
      \begin{tikzpicture}
        \begin{feynman}
          \vertex (tp);
          \vertex [right=1in of tp] (tpp);
          \diagram* {
          (tp) -- (tpp),
          (tp) --[loop, out=090, in=150, min distance=1cm] tp,
          (tp) --[loop, out=210, in=270, min distance=1cm] tp,
          (tpp) --[loop, out=030, in=090, min distance=1cm] tpp,
          (tpp) --[loop, out=270, in=330, min distance=1cm] tpp,
          };
          \node[circle, fill=black, inner sep=1pt] at (tp);
          \node[circle, fill=black, inner sep=1pt] at (tpp);
          \node[left=0.75cm] at (tp) {$\tp$};
          \node[right=0.75cm] at (tpp) {$\tpp$};
        \end{feynman}
      \end{tikzpicture}
    \end{center}
     &
    \begin{center}
      $1\fact \times \binom{5}{1}^2 \times \left((4 - 1)\doublefact\right)^2 = 225$
    \end{center}
    \\
    \begin{center}
      \begin{tikzpicture}
        \begin{feynman}
          \vertex (tp);
          \vertex [right=1in of tp] (tpp);
          \diagram* {
          (tp)  -- (tpp),
          (tp)  -- [quarter left]  (tpp),
          (tp)  -- [quarter right] (tpp),
          (tp)  -- [loop, out=135, in=225, min distance=1cm] tp,
          (tpp) -- [loop, out=045, in=315, min distance=1cm] tpp,
          };
          \node[circle, fill=black, inner sep=1pt] at (tp);
          \node[circle, fill=black, inner sep=1pt] at (tpp);
          \node[left=0.75cm] at (tp) {$\tp$};
          \node[right=0.75cm] at (tpp) {$\tpp$};
        \end{feynman}
      \end{tikzpicture}
    \end{center}
     &
    \begin{center}
      $3\fact \times \binom{5}{3}^2 = 600$
    \end{center}
    \\
    \begin{center}
      \begin{tikzpicture}
        \begin{feynman}
          \vertex (tp);
          \vertex [right=1in of tp] (tpp);
          \diagram* {
          (tp)  -- (tpp),
          (tp)  -- [quarter left]  (tpp),
          (tp)  -- [quarter right] (tpp),
          (tp)  -- [half left]  (tpp),
          (tp)  -- [half right] (tpp),
          };
          \node[circle, fill=black, inner sep=1pt] at (tp);
          \node[circle, fill=black, inner sep=1pt] at (tpp);
          \node[left=0.75cm] at (tp) {$\tp$};
          \node[right=0.75cm] at (tpp) {$\tpp$};
        \end{feynman}
      \end{tikzpicture}
    \end{center}
     &
    \begin{center}
      $5\fact \times \binom{5}{5}^2 = 120$
    \end{center}
  \end{tabular}
  \caption{Feynman diagrams and their multiplicities for $\avg{x(\tp)^5 x(\tpp)^5}$.}
  \label{fig:39a1}
\end{figure}

Generalising to odd integers $n \geq 3$, there are $\half (n + 1)$ diagrams
for\\ $m = \{1, 3, \dots, n-2, n\}$ connections between the legs of the $\tp$
and $\tpp$ vertices.
With $m$ connections, there are $m\fact \binom{n}{m}^2$ ways to choose $m$
pairs of the legs of the $\tp$ and $\tpp$ vertices and $(n - m - 1)\doublefact$
ways to choose pairs of the remaining legs of each of the $\tp$ and $\tpp$
vertices.
Hence, each of the $\half (n + 1)$ diagrams with $m$ connections has multiplicity:
\begin{equation*}
  m\fact \left( \binom{n}{m} (n - m - 1)\doublefact \right)^2
\end{equation*}
Define $n = 2j + 1$ and $m = 2k + 1$ where $j \in \N$, $k \in \{0, 1, \dots, j
  - 1, j\}$.
The average is:
\begin{align*}
   & \avg{\exp\left(-\epsilon\ihbar\int_0^t x(\tp)^{2j + 1} \,\dtp \right)}
  \approx 1
  - \frac{1}{2}\epsilon^2\frac{1}{\hbar^2}
  \int_0^t\dtp \int_0^t\dtpp
  \avg{x(\tp)^{2j + 1} x(\tpp)^{2j + 1}}
  \\[3ex]
   & \qquad \approx 1
  - \frac{1}{2}\epsilon^2\frac{1}{\hbar^2}\Biggl(\
  \sum_{k=0}^{j}
  (2k + 1)\fact \left( \binom{2j + 1}{2k + 1} \left(2(j - k) - 1\right)\doublefact \right)^2
  \dots
  \\[2ex]
   & \qquad\qquad \dots
  \int_0^t\dtp \int_0^t\dtpp\
  \left(iG(\tp, \tpp)\right)^{2k + 1}
  \left(G(\tp, \tp) G(\tpp, \tpp)\right)^{2(j - k)}
  \Biggr)
\end{align*}
