\section{Ignorance and uncertainty}

Probability theory conflates ignorance and uncertainty.
This section describes approaches to explicitly modelling them.

\subsection{Dempster-Shafer theory}

In Dempster-Shafer theory, uncertainty is quantified by two measures:

\begin{itemize}
  \item \textit{belief} (evidence that implies a proposition); and
  \item \textit{plausibility} (evidence that is consistent with a proposition).
\end{itemize}

\begin{dfn}
  [Belief and plausibility measures]
  \label{def:4:BeliefPlausibility}
  Let $W$ be a set of possible worlds and $A \subseteq W$ be a proposition.
  A mass function $m : 2^W \to [0, 1]$ generates:
  \begin{itemize}
    \item a belief measure $\bel : 2^W \to [0, 1]$ such that:
          \begin{equation}
            \label{eqn:4:Belief}
            \bel(A) =
            \sum_{B \subseteq W : B \subseteq A} m(B)
            \ \forall\
            A \subseteq W
          \end{equation}
    \item a plausibility measure $\pl : 2^W \to [0, 1]$ such that:
          \begin{equation}
            \label{eqn:4:Plausibility}
            \pl(A) =
            \sum_{B \subseteq W : B \cap A \neq \emptyset} m(B)
            \ \forall\
            A \subseteq W
          \end{equation}
  \end{itemize}
\end{dfn}

I.e., for a proposition $A \subseteq W$:

\begin{itemize}
  \item the \textit{belief} in $A$ is the sum of the masses of the subsets of
        $W$ that are subsets of $A$, i.e., of the evidence that implies $A$; and
  \item the \textit{plausibility} of $A$ is the sum of the masses of the subsets
        of $W$ that intersect $A$, i.e., of the evidence that is consistent with
        $A$.
\end{itemize}

\begin{thm}
  [Plausibility is greater than or equal to belief]
  Let $W$ be a set of possible worlds and $A \subseteq W$ be a proposition.
  \begin{equation}
    \label{eqn:4:PlausibilityGreaterThanBelief}
    \bel(A) \leq \pl(A) \ \forall\ A \subseteq W
  \end{equation}
  \begin{proof}
    $B \neq \emptyset,\ B \subseteq A \ \implies \ B \cap A \neq \emptyset$,
    i.e., a summation term in equation~\ref{eqn:4:Belief} is also a summation
    term in equation~\ref{eqn:4:Plausibility}.
    $m$ is non-negative, hence equation~\ref{eqn:4:PlausibilityGreaterThanBelief}.
  \end{proof}
\end{thm}

\begin{thm}
  [Dual relationship]
  \label{thm:4:DualRelationship}
  Let $W$ be a set of possible worlds and $A \subseteq W$ be a proposition.
  \begin{equation}
    \label{eqn:4:DualRelationship}
    \pl(A) = 1 - \bel{A^{c}} \ \forall\ A \subseteq W
  \end{equation}
  \begin{proof}
    $B \cap A \neq \emptyset \ \iff\  B \not\subseteq A^{c}$ and
    $\sum_{B \subseteq W} m(B) = 1$, hence:
    \begin{align*}
      \pl(A)
       & = \sum_{B \subseteq W : B \cap A \neq \emptyset} m(B)
      \ =\ \sum_{B \subseteq W : B \not\subseteq A^{c}} m(B)   \\[3ex]
       & = 1 - \sum_{B \subseteq W : B \subseteq A^{c}} m(B)
      \ =\ 1 - \bel(A^{c})
    \end{align*}
  \end{proof}
\end{thm}

\begin{thm}
  [Relation to probability theory]
  If $m : 2^W \to [0, 1]$ is a mass function such that
  $\sum_{w \in W} m(\{w\}) = 1$, then:
  \begin{equation}
    \bel(A) = \pl(A) = P(A) = \sum_{w \in A} m(\{w\})
  \end{equation}
\end{thm}

I.e., if $m$ is non-zero only for singletons, then each piece of evidence
identifies a single possible world.
Generally, belief and plausibility measures do not satisfy
definition~\ref{def:2:ProbabilityMeasure}.

\begin{thm}
  [Belief measures are super-additive]
  \begin{equation}
    \label{eqn:4:BeliefSuperAdditive}
    \bel(A \cup B) \geq \bel(A) + \bel(B)
  \end{equation}

  \begin{proof}
    Let $W$ be a set of possible worlds and $A, B \subseteq W$ be propositions
    such that $A \cap B = \emptyset$.
    Without loss of generality, assume that $A \neq \emptyset$ and
    $B \neq \emptyset$.
    A proposition $C \subseteq W : C \subseteq A \cup B$ if and only if:
    \begin{enumerate}
      \item $C \subseteq A$;
      \item $C \subseteq B$; or
      \item $C = D \cup E$ where $D \neq \emptyset, D \subseteq A$ and
            $E \neq \emptyset, E \subseteq B$.
            \footnote{
              I.e., if $C$ is not a subset of $A$ or $B$, then $C$ is the union of
              sets $D$ and $E$ that are subsets of $A$ and $B$ respectively.
            }
    \end{enumerate}
    By definition~\ref{def:4:BeliefPlausibility}:
    \begin{align*}
      \bel(A \cup B)
       & = \sum_{C \subseteq W : C \subseteq A \cup B} m(C)              \\[3ex]
       & = \sum_{C \subseteq W : C \subseteq A} m(C)
      + \sum_{C \subseteq W : C \subseteq B} m(C)                        \\
       & \quad + \sum_{D \subseteq W : D \neq \emptyset, D \subseteq A}\
      \sum_{E \subseteq W : E \neq \emptyset, E \subseteq B}
      m(D \cup E)                                                        \\[3ex]
       & = \bel(A) + \bel(B)                                             \\
       & \quad + \sum_{D \subseteq W : D \neq \emptyset, D \subseteq A}\
      \sum_{E \subseteq W : E \neq \emptyset, E \subseteq B} m(D \cup E)
    \end{align*}
    $m$ is non-negative, hence equation~\ref{eqn:4:BeliefSuperAdditive}.
  \end{proof}
\end{thm}

\begin{thm}
  [Plausibility measures are sub-additive]
  \begin{equation}
    \label{eqn:4:PlausibilitySubAdditive}
    \pl(A \cup B) \leq \pl(A) + \pl(B)
  \end{equation}

  \begin{proof}
    Let $W$ be a set of possible worlds and $A, B \subseteq W$ be propositions
    such that $A \cap B = \emptyset$.
    Without loss of generality, assume that $A \neq \emptyset$ and
    $B \neq \emptyset$.
    A proposition $C \subseteq W : C \cap (A \cup B) \neq \emptyset$ if and only
    if:
    \begin{enumerate}
      \item $C \cap A \neq \emptyset$ and $C \cap B \neq \emptyset$;
      \item $C \cap A \neq \emptyset$ and $C \cap B = \emptyset$; or
      \item $C \cap A = \emptyset$ and $C \cap B \neq \emptyset$.
    \end{enumerate}
    By definition~\ref{def:4:BeliefPlausibility}:
    \begin{align*}
      \pl(A \cup B)
       & = \sum_{C \subseteq W : C \cap (A \cup B) \neq \emptyset} m(C)                    \\[3ex]
       & = \sum_{C \subseteq W : C \cap A \neq \emptyset, C \cap B \neq \emptyset} m(C)
      + \sum_{C \subseteq W : C \cap A \neq \emptyset, C \cap B = \emptyset} m(C)          \\
       & \quad + \sum_{C \subseteq W : C \cap A = \emptyset, C \cap B \neq \emptyset} m(C)
    \end{align*}
    Similarly:
    \begin{align*}
      \pl(A) + \pl(B)
       & =
      \sum_{C \subseteq W : C \cap A \neq \emptyset} m(C) +
      \sum_{C \subseteq W : C \cap B \neq \emptyset} m(C)
      \\[3ex]
       & =
      \sum_{C \subseteq W : C \cap A \neq \emptyset, C \cap B \neq \emptyset} m(C) +
      \sum_{C \subseteq W : C \cap A \neq \emptyset, C \cap B = \emptyset} m(C)
      \\
       & \quad +
      \sum_{C \subseteq W : C \cap A \neq \emptyset, C \cap B \neq \emptyset} m(C) +
      \sum_{C \subseteq W : C \cap A = \emptyset, C \cap B \neq \emptyset} m(C)
      \\[3ex]
       & =
      \pl(A \cup B) +
      \sum_{C \subseteq W : C \cap A \neq \emptyset, C \cap B \neq \emptyset} m(C)
    \end{align*}
    $m$ is non-negative, hence equation~\ref{eqn:4:PlausibilitySubAdditive}.
  \end{proof}
\end{thm}

Given one of $m$, $\bel$ or $\pl$, the other two can be derived, i.e., they
contain the same information (theorems~\ref{thm:4:DualRelationship} and
\ref{thm:4:MassFunctionBelief}).

\begin{thm}
  [The mass function in terms of a belief measure]
  \label{thm:4:MassFunctionBelief}
  Let $W$ be a set of possible worlds,
  $\bel : 2^W \to [0, 1]$ be a belief measure, and
  $A \subseteq W$ be a proposition.
  The mass function $m : 2^W \to [0, 1]$ is:
  \begin{equation}
    \label{eqn:4:MassFunctionBelief}
    m(A) = \sum_{B \subseteq A} (-1)^{|A - B|} \bel(B)
    \ \forall\
    A \subseteq W
  \end{equation}

  \begin{proof}
    By induction on $|A|$.
    In the case that $|A| = 1$, $A = \{w\} : w \in W$ and $\bel(A) = m(A)$.
    Suppose that equation~\ref{eqn:4:MassFunctionBelief} holds for $|A| \leq n$.
    By definition~\ref{def:4:BeliefPlausibility}, if $|A| = n + 1$:
    \begin{equation*}
      \bel(A) = \sum_{B \subseteq A} m(B) = m(A) + \sum_{B \subset A} m(B)
    \end{equation*}
    If $B \subset A$, then $|B| \leq n$.
    By the inductive hypothesis:
    \begin{equation*}
      \bel(A) = m(A) + \sum_{B \subset A} \sum_{C \subseteq B} (-1)^{|B - C|} \bel(C)
    \end{equation*}
    Therefore:
    \begin{align}
      \label{eqn:4:MassFunctionBelief1}
      m(A)
       & =
      \bel(A) -
      \sum_{B \subset A} \sum_{C \subseteq B} (-1)^{|B - C|} \bel(C)
      \nonumber \\[3ex]
       & =
      \bel(A) +
      \sum_{B \subset A} \sum_{C \subseteq B} (-1)^{|B - C| + 1} \bel(C)
      \nonumber \\[3ex]
       & =
      \bel(A) +
      \sum_{C \subset A} \bel(C) \sum_{B : C \subseteq B \subseteq A} (-1)^{|B - C| + 1}
    \end{align}

    We have that:
    \begin{itemize}
      \item $C \subseteq B \ \Rightarrow\  |B - C| + 1 = |B| - |C| + 1$; and
      \item $C \subseteq B \subset A \ \Rightarrow\  0 \leq |B| - |C| \leq |A| - |C| - 1$.
    \end{itemize}
    A set $B : C \subseteq B \subset A$ is generated by choosing $i$
    elements from $A \cap C^c$ and taking their union with $D$.
    There are $\binom{|A| - |C|}{i}$ ways to do this.
    Hence:
    \begin{align}
      \label{eqn:4:MassFunctionBelief2}
      \sum_{B : C \subseteq B \subset A} (-1)^{|B - C| + 1}
       & = \sum_{i = 0}^{|A| - |C| - 1} \binom{|A| - |C|}{i} (-1)^{i + 1} \nonumber   \\[3ex]
       & = - \sum_{i = 0}^{|A| - |C| - 1} \binom{|A| - |C|}{i} (-1)^{i} \nonumber     \\[3ex]
       & = (- 1)^{|A| - |C|} - \sum_{i = 0}^{|A| - |C|} \binom{|A| - |C|}{i} (-1)^{i}
    \end{align}
    By the binomial theorem $\sum_{k = 0}^{n} \binom{n}{k} r^{k} = (1 + r)^{n}$:
    \begin{equation}
      \label{eqn:4:MassFunctionBelief3}
      \sum_{i = 0}^{|A| - |C|} \binom{|A| - |C|}{i} (-1)^{i}
      = (1 + (- 1))^{|A| - |C|}
      = 0
    \end{equation}
    By substituting equation~\ref{eqn:4:MassFunctionBelief3} into
    equation~\ref{eqn:4:MassFunctionBelief2}:
    \begin{equation}
      \label{eqn:4:MassFunctionBelief4}
      \sum_{B : C \subseteq B \subset A} (-1)^{|B - C| + 1}
      = (- 1)^{|A| - |C|}
    \end{equation}
    By substituting equation~\ref{eqn:4:MassFunctionBelief4} into
    equation~\ref{eqn:4:MassFunctionBelief1}:
    \begin{align*}
      m(A)
       & = \bel(A) + \sum_{C \subset A} \bel(C) (- 1)^{|A| - |C|} \\[3ex]
       & = \sum_{C \subseteq A}  (- 1)^{|A| - |C|} \bel(C)
    \end{align*}
  \end{proof}
\end{thm}

\subsection{Imprecise probabilities}

An alternative but related approach is to represent beliefs by sets of
probability measures (\textit{credal sets}).

\begin{dfn}
  [Lower and upper probability measures]
  Let $\mathbb{P}(K) \subseteq \mathbb{P}$ be a closed convex set of probability
  measures $P : 2^W \to [0, 1]$ and $A \subseteq W$ be a proposition.
  The lower ($\underline{P}$) and upper ($\overline{P}$) probability measures are:
  \begin{equation}
    \underline{P}(A) = \min\ \{ P(A) : P \in \mathbb{P}(K) \}\,,\
    \overline{P}(A)  = \max\ \{ P(A) : P \in \mathbb{P}(K) \}
  \end{equation}
\end{dfn}

\begin{thm}
  [Upper probability measure of complement]
  \begin{align}
    \overline{P}(A^{c})
     & = \max\ \{ P(A^c) : P \in \mathbb{P}(K) \} \nonumber   \\
     & = \max\ \{ 1 - P(A) : P \in \mathbb{P}(K) \} \nonumber \\
     & = 1 - \min\ \{ P(A) : P \in \mathbb{P}(K) \} \nonumber \\
     & = 1 - \underline{P}(A)
  \end{align}
\end{thm}

Belief and plausibility measures in Dempster-Shafer theory are special cases of
lower and upper probability measures, respectively.

\begin{thm}
  [Relations to belief and plausibility measures]
  Let $W$ be a set of possible worlds, $A \subseteq W$ be a proposition,
  $\bel : 2^W \to [0, 1]$ be a belief measure, and
  $K = \{ P(A) \geq \bel(A) \mid A \subseteq W \}$.
  \begin{equation}
    \bel(A) = \underline{P}(A)\,,\
    \pl(A) = \overline{P}(A)
    \ \forall\ A \subseteq W
  \end{equation}
  \begin{proof}
    In two parts:
    \begin{enumerate}
      \item $\bel(A) \leq P(A) \leq \pl(A) \ \forall\ P \in \mathbb{P}(K), A \subseteq W$
    \end{enumerate}

    By the definition of $K$,
    $P(A) \geq \bel(A)$ and $\bel(A^c) \leq P(A^c) \ \forall\ P \in \mathbb{P}(K)$.
    Hence, $1 - P(A^c) \leq 1 - \bel(A^c)$ and $P(A) \leq \pl(A)$.

    \begin{enumerate}
      \setcounter{enumi}{1}
      \item $\exists\ P \in \mathbb{P}(K) : P(A) = \bel(A) \ \forall\ A \subseteq W$
    \end{enumerate}

    For every $B \subseteq W$, choose a possible world $w_B \in B$ such that
    for a given $A \subseteq W$,  $B \not\subseteq A \Rightarrow w_B \in A^c$.
    Define $P$ in terms of the mass function $m$ of $\bel$:
    \begin{equation*}
      P(w) = \sum_{B \subseteq W : w_B = w} m(B)
    \end{equation*}
    $m(B)$ is non-zero only for $w_b$, so
    $\sum_{w \in W} P(w) = \sum_{B \subseteq W} m(B) = 1$.

    $P \in \mathbb{P}(K)$ because:
    \begin{align*}
      \bel(C)
       & = \sum_{B \subseteq W : B \subseteq C} m(B) \ \forall\ C \subseteq W
      = \sum_{w \in C} \sum_{B \subseteq W : B \subseteq C, w_B = w} m(B)     \\
       & \leq \sum_{w \in C} \sum_{B \subseteq W : w_B = w} m(B)
      = \sum_{w \in C} P(w)
      = P(C)
    \end{align*}
    Similarly:
    \begin{align*}
      \bel(A)
       & = \sum_{B \subseteq W : B \subseteq A} m(B)
      = \sum_{w \in A} \sum_{B \subseteq W : B \subseteq A, w_B = w} m(B) \\
       & = \sum_{w \in A} \sum_{B \subseteq W : w_B = w} m(B)
      = \sum_{w \in A} P(w)
      = P(A)
    \end{align*}
  \end{proof}
\end{thm}

A mass function assigns `weights' to pieces of evidence (sets of possible worlds).
The definition of conditional probability (\ref{def:2:ConditionalProbability})
can be generalised to mass functions.

\begin{dfn}
  [Posterior probability given a mass function]
  \label{def:4:PosteriorProbability}
  Let $W$ be a set of possible worlds, $A, B \subseteq W$ be propositions,
  $P : 2^W \to [0, 1]$ be a prior probability distribution, and
  $m : 2^W \to [0, 1]$ be a mass function.
  The posterior probability of $A$ given $m$ is:
  \begin{equation}
    P(A \mid m) = \sum_{B \subseteq W} P(A \mid B) m(B)
  \end{equation}
  $m(B) > 0 \Rightarrow P(B) > 0$, otherwise $P(A \mid m)$ is undefined.
  The posterior probability of $w$ given $m$ is:
  \begin{equation}
    P(w \mid m) = P(w) \sum_{B \subseteq W : w \in B} \frac{m(B)}{P(B)}
  \end{equation}
\end{dfn}

\begin{thm}
  [Relations to belief and plausibility measures]
  \begin{equation}
    \bel(A) \leq P(A \mid m) \leq \pl(A) \ \forall\ A \subseteq W
  \end{equation}
  \begin{proof}
    By definition~\ref{def:4:PosteriorProbability} and
    $B \subseteq A \Rightarrow P(A \mid B) = 1$:
    \begin{align*}
      P(A \mid m)
       & = \sum_{B \subseteq W} P(A \mid B) m(B)                  \\[1.5ex]
       & = \sum_{B \subseteq W : B \subseteq A} P(A \mid B) m(B)
      + \sum_{B \subseteq W : B \not\subseteq A} P(A \mid B) m(B) \\[1.5ex]
       & = \sum_{B \subseteq W : B \subseteq A} m(B)
      + \sum_{B \subseteq W : B \not\subseteq A} P(A \mid B) m(B) \\[1.5ex]
       & \geq \sum_{B \subseteq W : B \subseteq A} m(B) = \bel(A) \\[3ex]
      P(A \mid m)
       & = 1 - P(A^c \mid m)                                      \\[1.5ex]
       & \leq 1 - \bel(A^c) = \pl(A)
    \end{align*}
  \end{proof}
\end{thm}

If the prior probability distribution is uniform, then
definition~\ref{def:4:PosteriorProbability} is:
\begin{equation*}
  P(w \mid m) = \sum_{B \subseteq W : w \in B} \frac{m(B)}{\card{B}}
\end{equation*}
I.e., the posterior probability distribution redistributes the mass values
associated with non-singleton sets uniformly to the singleton sets of their
elements.
This is called the \textit{pignistic distribution} of a mass function.

\subsection{Combination in Dempster-Shafer theory}

There is a natural way to represent the combination of evidence from independent
sources (mass functions) in Dempster-Shafer theory.

\begin{dfn}
  [Dempster's rule of combination]
  \label{dfn:4:DempstersRuleOfCombination}
  Let $W$ be a set of possible worlds, $A \subseteq W$ be a proposition, and
  $m_1, m_2 : 2^W \to [0, 1]$ be mass functions.
  The combination of $m_1$ and $m_2$ is:
  \begin{equation}
    \label{eqn:4:DempstersRuleOfCombination}
    m_1 \oplus m_2 (A)
    = \frac{
      \sum_{(B, C) : B \cap C = A} m_1(B) m_2(C)
    }{
      1 - \sum_{(B, C) : B \cap C \neq \emptyset} m_1(B) m_2(C)
    } \ \forall\ A \subseteq W
  \end{equation}
\end{dfn}

Shafer proposed a definition of conditional belief and plausibility using
definition~\ref{dfn:4:DempstersRuleOfCombination}: a proposition $B$ can be
interpreted as evidence (represented by a mass function $m_B$).
Belief and plausibility measures can be conditioned on $B$ by combining the
generating mass function $m$ with $m_B$ according to
definition~\ref{dfn:4:DempstersRuleOfCombination}.

\begin{dfn}
  [Conditional belief and plausibility]
  \label{dfn:4:ConditionalBeliefPlausibility}
  Let $W$ be a set of possible worlds, $A, B \subseteq W$ be propositions, and
  $\bel,\ \pl,\ m : 2^W \to [0, 1]$ be belief and plausibility measures and a
  mass function.
  The conditional belief and plausibility of $A$ given $B$ are:
  \begin{align}
    \label{eqn:4:ConditionalBelief}
    \bel(A \mid B)
     & = \sum_{C : C \subseteq A} m \oplus m_B(C)           \\[2ex]
    \label{eqn:4:ConditionalPlausibility}
    \pl(A \mid B)
     & = \sum_{C : C \cap A \neq \emptyset} m \oplus m_B(C)
  \end{align}
  where $m_B(B) = 1$, $m_B(C) = 0\ \forall\ C \neq B$, and $\pl(B) > 0$.
\end{dfn}

\begin{thm}
  [Relation to plausibility]
  \label{thm:4:RelationToPlausibility}
  Let $W$ be a set of possible worlds, $A, B \subseteq W$ be propositions, and
  $\pl : 2^W \to [0, 1]$ be a plausibility measure.
  \begin{equation}
    \label{eqn:4:RelationToPlausibility}
    \pl(A \mid B) = \frac{\pl(A \cap B)}{\pl(B)}
  \end{equation}
  \begin{proof}
    By definitions~\ref{dfn:4:ConditionalBeliefPlausibility} and
    \ref{dfn:4:DempstersRuleOfCombination} and $\sum_D m(D) = 1$:
    \begin{align*}
      \pl(A \mid B)
       & = \sum_{C : C \cap A \neq \emptyset} m \oplus m_B(C)
      = \sum_{C : C \cap A \neq \emptyset} \frac{\sum_{D : D \cap B = C} m(D)}{1 - \sum_{D : D \cap B = \emptyset} m(D)}
      \\[3ex]
       & = \frac{\sum_{C : C \cap A \neq \emptyset} \sum_{D : D \cap B = C} m(D)}{1 - \sum_{D : D \cap B = \emptyset} m(D)}
      =  \frac{\sum_{C : C \cap A \neq \emptyset} \sum_{D : D \cap B = C} m(D)}{\sum_{D : D \cap B \neq \emptyset} m(D)}
      \\[3ex]
       & = \frac{\sum_{D : (D \cap B) \cap A \neq \emptyset} m(D)}{\sum_{D : D \cap B \neq \emptyset} m(D)}
      = \frac{\sum_{D : D \cap (B \cap A) \neq \emptyset} m(D)}{\sum_{D : D \cap B \neq \emptyset} m(D)}
      = \frac{\pl(A \cap B)}{\pl(B)}
    \end{align*}
  \end{proof}
\end{thm}

\begin{thm}
  [Relation to belief]
  Let $W$ be a set of possible worlds, $A, B \subseteq W$ be propositions, and
  $\bel : 2^W \to [0, 1]$ be a belief measure.
  \begin{equation}
    \label{eqn:4:RelationToBelief}
    \bel(A \mid B) = \frac{\bel(A \cup B^c) - \bel(B^c)}{1 - \bel(B^c)}
  \end{equation}
  \begin{proof}
    By theorems~\ref{thm:4:DualRelationship} and
    \ref{thm:4:RelationToPlausibility}:
    \begin{align*}
      \bel(A \mid B)
       & = 1 - \pl(A^c \mid B)
      = 1 - \frac{\pl(A^c \cap B)}{\pl(B)}
      = \frac{\pl(B) - \pl(A^c \cap B)}{\pl(B)}
      \\[3ex]
       & = \frac{1 - \bel(B^c) - (1 - \bel((A^c \cap B)^c))}{1 - \bel(B^c)}
      = \frac{\bel(A \cup B^c) - \bel(B^c)}{1 - \bel(B^c)}
    \end{align*}
  \end{proof}
\end{thm}

Definition~\ref{dfn:4:DempstersRuleOfCombination} produces counterintuitive
results if the mass functions are highly inconsistent, i.e., the denominator
$1 - \sum_{(B, C) : B \cap C \neq \emptyset} m_1(B) m_2(C) \approx 0$.