\section{Probabilistic reasoning}

Classical logic is based on propositions and rules that are known with certainty.
This is generally impossible.
Probabilistic reasoning is based on uncertain knowledge.
There are generally multiple probability distributions that are consistent with
a knowledge-base.

\subsection{Probabilistic knowledge-based reasoning}

One approach to probabilistic reasoning is to identify the set of probability
distributions $\mathbb{P}(K)$ that are consistent with a knowledge-base $K$,
then select a probability distribution based on some principle.

\begin{dfn}
  [Probabilistic knowledge-base]
  A probabilistic knowledge-base $K$ is a set of linear equations on $P$:
  \begin{equation}
    K = \left\{
    \sum_{i = 1}^{n_j}a_{ij} P(A_{ij}) = b_j
    \ :\
    j = 1 .. m
    \ \land \
    A_{ij}\subseteq W
    \ \land \
    a_{ij},b_k \in\mathbb{R}
    \right\}
  \end{equation}
\end{dfn}

One principle is to select the distribution that has minimal information.
Entropy is a measure of the information content of a probability distribution.

\begin{dfn}
  [Entropy]
  Let $W = \{w_i \mid i \in 1 .. n\}$ and $P(w_i) = p_i : i \in 1 .. n$.
  Entropy is:
  \begin{equation}
    H(\{p_i \mid i \in 1 .. n\}) = \sum_{i = 1}^{n} - p_i \log_2(p_i)
  \end{equation}
\end{dfn}

If the knowledge-base is linear, then there is a single probability distribution
with maximum entropy (theorem~\ref{thm:3:MaximumEntropyDistribution}).

\begin{thm}
  [Maximum entropy distribution]
  \label{thm:3:MaximumEntropyDistribution}
  Let $W=\{w_i \mid i \in 1 .. n\}$.
  The maximum entropy distribution in $\mathbb{P}$ is $P(w_i) = \frac{1}{n}$.
  \begin{proof}
    Let $P(w_i) = p_i : i\in 1 .. n$.
    Without loss of generality, let $p_n = 1 - \sum_{i = 1}^{n - 1} p_i$ such
    that $H$ is a function of
    $\{p_i \mid i \in 1 .. n - 1\}$.

    $H$ is maximal when $\frac{\partial H}{\partial p_i} = 0 : i \in 1 .. n - 1$.
    By the product and chain rules:
    \begin{align}
      \frac{\partial H}{\partial p_i}
       & = \frac{\partial}{\partial p_i}(- p_i \log_2(p_i))
      - \frac{\partial}{\partial p_i}(- p_n \log_2(p_n))
      \label{eqn:3:MaximumEntropyDistribution1}             \\
       & = -\log_2(p_i) \frac{\partial p_i}{\partial p_i}
      - p_i \frac{\partial \log_2(p_i)}{\partial p_i}
      + \log_2(p_n) \frac{\partial p_n}{\partial p_i}
      + p_n \frac{\partial \log_2(p_n)}{\partial p_i}
      \label{eqn:3:MaximumEntropyDistribution2}
    \end{align}
    By the properties of logarithms:
    \begin{align}
      \frac{\partial \log_2(p_i)}{\partial p_i}
       & = \frac{1}{\ln 2} \frac{\partial \ln p_i}{\partial p_i}
      = \frac{1}{p_i \ln 2}
      \label{eqn:3:MaximumEntropyDistribution3}                  \\
      \frac{\partial \log_2(p_n)}{\partial p_i}
       & = \frac{1}{\ln 2} \frac{\partial \ln p_n}{\partial p_i}
      = \frac{1}{\ln 2} \frac{\partial p_n}{\partial p_i} \frac{\partial \ln p_n}{p_n}
      = - \frac{1}{p_n \ln 2}
      \label{eqn:3:MaximumEntropyDistribution4}
    \end{align}
    Substituting equations~\ref{eqn:3:MaximumEntropyDistribution3} and
    \ref{eqn:3:MaximumEntropyDistribution4} into
    equation~\ref{eqn:3:MaximumEntropyDistribution2} yields:
    \begin{equation}
      \frac{\partial H}{\partial p_i} = - \log_2(p_i) - \log_2(p_n) = 0
    \end{equation}
    That is, $\log_2(p_i) = \log_2(p_n)$ and $p_i = p_n : i \in \{ 1 .. n - 1 \}$.
    Since $\sum_{i = 1}^{n} p_i = 1$ and $p_n = 1 - \sum_{i = 1}^{n - 1} p_i$,
    $p_i = \frac{1}{n}$.
  \end{proof}
\end{thm}

Another principle is to define a uniform distribution over the set of
probability distributions, then select the expected value of that distribution,
i.e., its \textit{centre of mass} (definition~\ref{def:3:CentreOfMass}).

\begin{dfn}
  [Centre of mass]
  \label{def:3:CentreOfMass}
  Let $\mathbb{P}(K)$ be a set of probability distributions that are consistent
  with a knowledge-base $K$. The centre of mass is:
  \begin{equation}
    P(w_i) = \frac{\int_{\mathbb{P}(K)} p_i \, d \mathbb{P}(K) }{\int_{\mathbb{P}(K)} d \mathbb{P}(K)}
  \end{equation}
\end{dfn}

\subsection{Probability logic}

Probability theory is not truth-functional.
Knowledge of $P(A)$ and $P(B)$ does not imply knowledge of, e.g., $P(A \cap B)$,
but it does imply upper and lower bounds
(theorems~\ref{thm:3:BoundsIntersection} and \ref{thm:3:BoundsUnion}).

\begin{thm}
  [Intersection bounds]
  \label{thm:3:BoundsIntersection}
  Let $A, B \subseteq W$.
  \begin{equation}
    \label{eqn:3:BoundsIntersection}
    \max(0, P(A) + P(B) - 1) \leq P(A \cap B) \leq \min(P(A), P(B))
  \end{equation}
  \begin{proof}
    By definition~\ref{def:2:ProbabilityMeasure}:
    \begin{align}
      P(A) & = P(A \cap B) + P(A \cap B^c) \label{eqn:3:BoundsIntersection1} \\
      P(B) & = P(A \cap B) + P(A^c \cap B) \label{eqn:3:BoundsIntersection2}
    \end{align}
    Since $P(A \cap B^c) \geq 0$ and $P(A^c \cap B) \geq 0$:
    \begin{equation}
      P(A \cap B) \leq \min(P(A), P(B)) \label{eqn:3:BoundsIntersection3}
    \end{equation}
    By theorem~\ref{thm:2:GeneralAdditivity}:
    \begin{equation}
      P(A \cap B) = P(A) + P(B) - P(A \cup B) \label{eqn:3:BoundsIntersection4}
    \end{equation}
    Since $P(A \cup B) \leq 1$ and $P(A \cap B) \geq 0$:
    \begin{equation}
      P(A \cap B) \geq \max(0, P(A) + P(B) - 1) \label{eqn:3:BoundsIntersection5}
    \end{equation}
  \end{proof}
\end{thm}

\begin{thm}
  [Union bounds]
  \label{thm:3:BoundsUnion}
  Let $A, B \subseteq W$.
  \begin{equation}
    \max(P(A), P(B)) \leq P(A \cup B) \leq \min(P(A) + P(B), 1)
  \end{equation}
  \begin{proof}
    By theorems~\ref{thm:2:GeneralAdditivity} and \ref{thm:3:BoundsIntersection}:
    \begin{equation}
      P(A \cup B) \geq P(A) + P(B) - \min(P(A), P(B)) = \max(P(A), P(B))
      \label{eqn:3:BoundsUnion1}
    \end{equation}
    By theorem~\ref{thm:3:BoundsIntersection}:
    \begin{align}
      P(A \cup B) & \leq P(A) + P(B) - \max(0, P(A) + P(B) - 1)          \\
                  & \leq \min(P(A) + P(B), 1) \label{eqn:3:BoundsUnion3}
    \end{align}
  \end{proof}
\end{thm}

Theorems~\ref{thm:2:Complement}, \ref{thm:3:BoundsIntersection}, and
\ref{thm:3:BoundsUnion} define a truth-functional logic for probability
intervals.
For a proposition $A \subseteq W$, it can be used to infer upper and lower
bounds on $P(A)$, i.e., $P(A) \in [L(A),\ U(A)]$.

\begin{thm}
  [Complement rule]
  \begin{equation}
    P(A^c) \in [1 - U(A),\ 1 - L(A)]
  \end{equation}
\end{thm}

\begin{thm}
  [Intersection rule]
  \begin{equation}
    P(A \cap B) \in [\max(0, L(A) + L(B) - 1),\ \min(U(A), U(B))]
  \end{equation}
\end{thm}

\begin{thm}
  [Union rule]
  \begin{equation}
    P(A \cup B) \in [\max(L(A), L(B)),\ \min(U(A) + U(B),\ 1)]
  \end{equation}
\end{thm}

\begin{thm}
  [Conditional rule]
  \begin{equation}
    P(A) \in [L(A \mid B) L(B),\ (U(A \mid B) - 1) U(B) + 1]
  \end{equation}
  \begin{proof}
    By theorem~\ref{thm:2:TotalProbability} and
    definition~\ref{def:2:ProbabilityMeasure}, i.e., $P(A \mid B^c) \in [0, 1]$:
    \begin{itemize}
      \item If $P(A \mid B^c) = 0$, then $P(A) = P(A \mid B) P(B)$ (lower bound).
      \item If $P(A \mid B^c) = 1$, then $P(A) = (P(A \mid B) - 1) P(B) + 1$ (upper bound).
    \end{itemize}
  \end{proof}
\end{thm}

\begin{thm}
  [Jeffrey's rule]
  \begin{equation}
    P^\prime(A) \in [L(A \mid B) L(B),\ (U(A \mid B) - 1) U(B) + 1]
  \end{equation}
\end{thm}

The upper and lower bounds of this logic are wider than probability theory
constrains them to be.
This is because the logic does not account for logical dependencies between
propositions and their relations to definition~\ref{def:2:ProbabilityMeasure}.

\subsection{Bayesian networks}

\subsubsection{Joint and marginal probability distributions}

\begin{dfn}
  [Joint probability distribution]
  The joint probability distribution of
  ${X = \{ X_i : W \rightarrow \Omega \mid i \in 1 .. n \}}$ is:
  \begin{align}
    P(X)
     & = P \bigl( \bigwedge_{i = 1}^{n} X_i = x_i \bigr) \nonumber \\
     & = P(\{ w : X_1(w) = x_1, \ldots, X_n(w) = x_n \})
  \end{align}
\end{dfn}

If $|\{ X_i(w) : w \in W \}| = k_i$, then the joint probability distribution
has $k_i^n - 1$ values, i.e., the number of values grows exponentially with the
dimension of the random variables (the \textit{curse of dimensionality}).

\begin{dfn}
  [Marginal probability distribution]
  The marginal probability distribution of $X_i : W \rightarrow \Omega$ is:
  \begin{equation}
    P(X_i = x_i) = P(\{ w : X_i(w) = x_i \})
    \ \forall\
    i \in 1 .. n
  \end{equation}
\end{dfn}

\begin{thm}
  [Joint and marginal probability distributions]
  The joint and marginal probability distributions are related by:
  \begin{equation}
    P(X_i = x_i) =
    \sum_{j \neq i}
    \sum_{x_j}
    P \bigl( \bigwedge_{j = 1}^{n} X_j = x_j \bigr)
    \ \forall\ i \in 1 .. n
  \end{equation}
\end{thm}

I.e., the marginal probability distribution of $X_i$ is the sum of the joint
probability distributions of $X$ over all values $x_j$ where $j \neq i$.

\subsubsection{Independence}

By definition~\ref{def:3:Independence}, if $X$ are \textit{independent}, then
the joint probability distribution of $X$ is defined by its \textit{marginal}
distributions.
Then, the number of values grows only linearly with the dimension of the random
variables.

\begin{dfn}
  [Independence]
  \label{def:3:Independence}
  Let $X = \{ X_i \mid i \in 1 .. j \}$ be a set of random variables.
  $X$ are independent if:
  \begin{equation}
    P \bigl( \bigwedge_{i = 1}^{n} X_i = x_i \bigr)
    = \prod_{i = 1}^{n} P(X_i = x_i)
    \ \forall \
    x_i \in \{ X_i(w) : w \in W \}
  \end{equation}
\end{dfn}

\subsubsection{Conditional distributions}

We cannot generally assume that random variables are independent.
But some random variables are not directly dependent on others.
The formalization of this intuition helps to make probabilistic reasoning
computationally feasible.

\begin{dfn}
  [Conditional distribution]
  Let $X = \{ X_i \mid i \in 1 .. j \}$ be a set of random variables.
  Without loss of generality, let $\{ X_1, X_2 \} \in \mathbb{X}$ form a
  partition of $X$:
  \begin{align*}
    X_1 & = \{ X_i \mid i \in 1 .. k - 1 \} \\
    X_2 & = \{ X_i \mid i \in k .. j \}
  \end{align*}
  The conditional probability of $\bigwedge_{i = 1}^{k - 1} X_i = x_i$
  given $\bigwedge_{i = k}^{j} X_i = x_i$ is:
  \begin{equation}
    \label{eqn:3:ConditionalDistribution}
    P \bigl( \bigwedge_{i = 1}^{k - 1} X_i = x_i \mid \bigwedge_{i = k}^{j} X_i = x_i \bigr)
    = \frac{P(\bigwedge_{i = 1}^{j} X_i = x_i)}{P(\bigwedge_{i = k}^{j} X_i = x_i)}
  \end{equation}
  The denominator of equation~\ref{eqn:3:ConditionalDistribution} is the joint
  probability distribution:
  \begin{equation}
    P(\bigwedge_{i = k}^{j} X_i = x_i)
    = \sum_{x_1} \ldots \sum_{x_{j - 1}} P(\bigwedge_{l = 1}^{j} X_l = x_l)
  \end{equation}
\end{dfn}

I.e., if an agent receives information that $\bigwedge_{l = k}^{j} X_l = x_l$,
then to update its probabilities for the other random variables
$\{ X_i \mid i \in 1 .. k - 1 \}$, it evaluates the conditional probability of
the other random variables given the information.

\begin{dfn}
  [Conditional independence]
  Let $X = \{ X_i \mid i \in 1 .. j \}$ be a set of random variables.
  Without loss of generality, let $\{ X_1, X_2, X_3 \} \in \mathbb{X}$ form a
  partition of $X$:
  \begin{align*}
    X_1 & = \{ X_i \mid i \in 1 .. k - 1 \} \\
    X_2 & = \{ X_i \mid i \in k .. l - 1 \} \\
    X_3 & = \{ X_i \mid i \in l .. j \}
  \end{align*}
  Then $X_1$ are conditionally independent of $X_3$ given $X_2$ if:
  \begin{multline}
    P \bigl( \bigwedge_{i = 1}^{k - 1} X_i = x_i \mid \bigwedge_{i = k}^{j} X_i = x_i \bigr) \\
    = P \bigl( \bigwedge_{i = 1}^{k - 1} X_i = x_i \mid \bigwedge_{i = k}^{l - 1} X_i = x_i \bigr)
    \ \forall\
    x_i \in \{ X_i(w) : w \in W \}
  \end{multline}
\end{dfn}

I.e., if an agent knows the values of $\{ X_i \mid i \in k .. l - 1 \}$, then
it can ignore the values of $\{ X_i \mid i \in l .. j \}$ when updating its
probabilities for $\{ X_i \mid i \in 1 .. k - 1 \}$.

\subsubsection{Bayesian networks}

A Bayesian network is a graphical model of probabilistic reasoning with multiple
random variables.
It is a compromise between independence and dependence: it assumes
\textit{independence where possible} and \textit{dependency where necessary}.

\begin{dfn}
  [Directed graph]
  A directed graph is an ordered pair $(V, E)$ where
  $V = \{ v_i \mid i \in 1 .. j \}$ is a set of vertices and $E$ is a binary
  relation on $V$ that defines a set of edges.
\end{dfn}

\begin{dfn}
  [Directed acyclic graph]
  A directed graph $(V, E)$ is acyclic if there is no sequence of vertices
  $v_i \ldots v_k$ where $v_i = v_k$ and $(v_l, v_{l + 1}) \in E$ for all
  $l \in 1 .. k - 1$.
\end{dfn}

\begin{dfn}
  [Bayesian network]
  \label{def:3:BayesianNetwork}
  A Bayesian network is:
  \begin{itemize}
    \item a directed acyclic graph $(V, E)$ where each vertex $v_i \in V$ is a
          random variable $X_i : W \rightarrow \Omega$ and $(X_i, X_j) \in E$
          only if $j < i$; and
    \item a joint probability distribution on $\{ X_i \mid i \in 1 .. k \}$
          where:
          \begin{equation}
            P(X_i \mid \{ X_j \mid j \in 1 .. i - 1 \})
            = P(X_i \mid \Pi(X_i))
          \end{equation}
          and $\Pi(X_i) = \{ X_k : (X_k, X_i) \in E \}$ is the set of parent
          vertices of $X_i$.
  \end{itemize}
\end{dfn}

For a Bayesian network, we assume that $X_i$ is conditionally independent of its
\textit{indirect} causes $\{ X_j \mid j \in 1 .. i - 1 \} - \Pi(X_i)$ given its
\textit{direct} causes $\Pi(X_i)$.
The joint probability distribution of $X$ is determined by its conditional
distributions $P(X_i \mid \Pi(X_i)) \ \forall\ i \in 1 .. k$
(theorem~\ref{thm:3:BayesianNetworkJPD}).

\begin{thm}
  [Joint probability distribution of a Bayesian network]
  \label{thm:3:BayesianNetworkJPD}
  \begin{equation}
    P(X) = \prod_{i = 1}^{k} P(X_i \mid \Pi(X_i))
  \end{equation}
  \begin{proof}
    Trivially:
    \begin{equation}
      P(\{ X_i \mid i \in 1 .. k \})
      = \prod_{i = 1}^{k} P(X_i \mid \{ X_j \mid j \in 1 .. i - 1 \})
    \end{equation}
    By the conditional independence assumptions of
    definition~\ref{def:3:BayesianNetwork}:
    \begin{equation}
      \prod_{i = 1}^{k} P(X_i \mid \{ X_j \mid j \in 1 .. i - 1 \})
      = \prod_{i = 1}^{k} P(X_i \mid \Pi(X_i))
    \end{equation}
  \end{proof}
\end{thm}

\subsubsection{Numbers of probability values}

For $n$ binary random variables, the number of probability values in the joint
probability distribution of:
\begin{itemize}
  \item a fully dependent model is $2^n - 1$;
  \item a fully independent model is $2n - 1$; and
  \item a Bayesian network is $\sum_{i = 1}^{n} 2^{|\Pi(X_i)|}$.
\end{itemize}

\begin{thm}
  [Number of probability values for a Bayesian network]
  For $n$ binary random variables, the joint probability distribution of a
  Bayesian network has between $n$ and $2^n - 1$ values.
  \begin{proof}
    The minimum of $|\Pi(X_i)|$ is 0, in which case:
    \begin{equation}
      \sum_{i = 1}^{n} 2^{|\Pi(X_i)|} = \sum_{i = 1}^{n} 2^0 = n
    \end{equation}
    The maximum of $|\Pi(X_i)|$ is $i - 1$, in which case:
    \begin{equation}
      \sum_{i = 1}^{n} 2^{|\Pi(X_i)|}
      = \sum_{i = 1}^{n} 2^{i - 1}
      = \sum_{i = 0}^{n - 1} 2^i
      = \frac{1 (1 - 2^n)}{1 - 2}
      = 2^n - 1
    \end{equation}
  \end{proof}
\end{thm}

For $n$ random variables, where the random variable $X_i$ has $k_i$ possible
values, i.e., $|\{ X_i(w) : w \in W \}| = k_i \ \forall\ i \in 1 .. n$, the
number of values in:
\begin{itemize}
  \item the conditional distributions $P(X_i \mid \Pi(X_i))$ is
        $(k_i - 1) \prod_{X_j \in \Pi(X_i)} k_j$; and
  \item the joint probability distribution of a Bayesian network is

        $\sum_{i = 1}^{n} (k_i - 1) \prod_{X_j \in \Pi(X_i)} k_j$.
\end{itemize}

Generally, the number of values for a Bayesian network is inversely proportional
to the numbers of direct causes of the random variables.