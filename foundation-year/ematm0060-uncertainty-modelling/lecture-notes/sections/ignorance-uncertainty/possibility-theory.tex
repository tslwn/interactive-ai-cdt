\subsection{Possibility theory}

Dempster-Shafer theory can be computationally expensive.
For $n$ possible worlds, the mass function has $2^n - 2$ values and, in the
worst case, Dempster's rule of combination requires $O(2^{2n})$ operations.

\emph{Possibility theory} is a special case of Dempster-Shafer theory that is
more computationally efficient but still differentiates between ignorance and
uncertainty.
Possibility and necessity measures are maxitive and minimal uncertainty
measures.

\begin{dfn}[Possibility and necessity measures]
  \label{def:4:PossibilityNecessity}
  A possibility measure is a function $P : 2^W \to [0, 1]$ such that:
  \begin{itemize}
    \item $\poss(W) = 1$ and $\poss(\emptyset) = 0$
    \item $\poss(A \cup B) = \max(\poss(A), \poss(B))$
  \end{itemize}
  A necessity measure is the dual measure $\necs(A) = 1 - \poss(A^c)$ where:
  \begin{itemize}
    \item $\necs(W) = 1$ and $\necs(\emptyset) = 0$
    \item $\necs(A \cup B) = \min(\necs(A), \necs(B))$
  \end{itemize}
\end{dfn}

The definitions of possibility and necessity measures have notable consequences
for a proposition $A \subseteq W$.
\begin{itemize}
  \item Either $A$ or its negation is absolutely possible:
        \begin{equation}
          \poss(W) = 1 = \poss(A \cup A^c) = \max(\poss(A), \poss(A^c))
        \end{equation}
  \item Either $A$ or its negation is absolutely not necessary:
        \begin{equation}
          \necs(\emptyset) = 0 = \necs(A \cap A^c) = \min(\necs(A), \necs(A^c))
        \end{equation}
  \item If $A$ is somewhat not necessary, then $A$ is absolutely not possible:
        \begin{equation}
          \necs(A) > 0 \implies \poss(A) = 0
        \end{equation}
  \item If $A$ is somewhat not possible, then $A$ is absolutely not necessary:
        \begin{equation}
          \poss(A) < 1 \implies \necs(A) = 0
        \end{equation}
  \item The `ignorance' associated with $A$ is equal to its possibility or the
        possibility of its complement:
        \begin{equation}
          \poss(A) - \necs(A) =
          \begin{cases}
            \poss(A)   & \text{if } \poss(A) < 1
            \\
            \poss(A^c) & \text{if } \necs(A) > 0
          \end{cases}
        \end{equation}
  \item The possibility measure $\poss(A)$ is uniquely determined by the
        possibility values of singleton sets (the \emph{possibility distribution}):
        \begin{equation}
          \poss(A) = \max_{w \in A} \poss(\{w\}) \ \forall\ A \subseteq W
        \end{equation}
        i.e., the number of values in the possibility distribution grows
        linearly with the number of possible worlds.
  \item At least one possible world must be maximally possible:
        \begin{equation}
          \max_{w \in W} \poss(\{w\}) = 1
        \end{equation}
\end{itemize}

\subsubsection{Relation to Dempster-Shafer theory}

Necessity and possibility measures are Dempster-Shafer belief and plausibility
measures for a particular form of the mass function.

\begin{thm}
  \label{thm:4:RelationToDempsterShafer}
  Let $m$ be a mass function on $2^W$ such that $\sum_{i = 1}^k m(F_i) = 1$ where
  $F_i \subseteq F_{i + 1} : i = 1 .. k - 1$.
  The belief and plausibility measures of $m$ are necessity and possibility
  measures respectively.
  \begin{proof}
    By definitions~\ref{def:4:BeliefPlausibility} and
    \ref{def:4:PossibilityNecessity}, if $\pl$ is a possibility measure, then
    $\bel$ is a necessity measure.
    Also, because $\pl(\emptyset) = 0$ and $\pl(W) = 1$, if $\pl(A \cup B) =
      \max(\pl(A), \pl(B))$, then theorem~\ref{thm:4:RelationToDempsterShafer} is
    true.

    Let $F_t$ be the smallest set in $F_1 \ldots F_k$ such that $A \cap F_t \neq
      \emptyset$.
    Then:
    \begin{equation*}
      \pl(A) = \sum_{C : C \cap A \neq \emptyset} m(C) = \sum_{i = t}^k m(F_i)
    \end{equation*}
    Likewise, let $F_s$ be the smallest set in $F_1 \ldots F_k$ such that $B \cap
      F_s \neq \emptyset$.
    Then:
    \begin{equation*}
      \pl(B) = \sum_{i = s}^k m(F_i)
    \end{equation*}
    and $F_{\min(s, t)}$ is the smallest set in $F_1 \ldots F_k$
    such that $A \cup B \neq \emptyset$.
    Hence:
    \begin{align*}
      \pl(A \cup B)
       & = \sum_{i = \min(s, t)}^k m(F_i)
      = \max(\sum_{i = t}^k m(F_i), \sum_{i = s}^k m(F_i))
      \\[2ex]
       & = \max(\pl(A), \pl(B))
    \end{align*}
  \end{proof}
\end{thm}

Given a possibility distribution $\poss$, a unique mass function $m$ can be
found such that $\poss(A) = \pl(A)$ for its associated plausibility measure.
Order the possible worlds $w_1 \ldots w_n$ such that $\poss(\{w_i\}) \geq
  \poss(\{w_{i + 1}\})$.
Then:
\begin{equation*}
  m(\{w_j \mid 1 .. i \}) :=
  \begin{cases}
    \poss(\{w_i\}) - \poss(\{w_{i + 1}\}) & \text{if } i < n
    \\
    \poss(\{w_n\})                        & \text{if } i = n
  \end{cases}
\end{equation*}
I.e., necessity and possibility measures are Dempster-Shafer belief and
plausibility measures whose mass function is non-zero only on a nested sequence
of sets.
