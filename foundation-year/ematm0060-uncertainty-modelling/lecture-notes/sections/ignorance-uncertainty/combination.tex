\subsection{Combination in Dempster-Shafer theory}

There is a natural way to represent the combination of evidence from
independent sources (mass functions) in Dempster-Shafer theory.

\begin{dfn}[Dempster's rule of combination]
  \label{dfn:4:DempstersRuleOfCombination}
  Let $W$ be a set of possible worlds, $A \subseteq W$ be a proposition, and
  $m_1, m_2 : 2^W \to [0, 1]$ be mass functions.
  The combination of $m_1$ and $m_2$ is:
  \begin{equation}
    \label{eqn:4:DempstersRuleOfCombination}
    m_1 \oplus m_2 (A)
    = \frac{
      \sum_{(B, C) : B \cap C = A} m_1(B) m_2(C)
    }{
      1 - \sum_{(B, C) : B \cap C \neq \emptyset} m_1(B) m_2(C)
    }
  \end{equation}
\end{dfn}

Shafer proposed a definition of conditional belief and plausibility using
definition~\ref{dfn:4:DempstersRuleOfCombination}: a proposition $B$ can be
interpreted as evidence (represented by a mass function $m_B$).
Belief and plausibility measures can be conditioned on $B$ by combining the
generating mass function $m$ with $m_B$ according to
definition~\ref{dfn:4:DempstersRuleOfCombination}.

\begin{dfn}[Conditional belief and plausibility]
  \label{dfn:4:ConditionalBeliefPlausibility}
  Let $W$ be a set of possible worlds, $A, B \subseteq W$ be propositions, and
  $\bel,\ \pl,\ m : 2^W \to [0, 1]$ be belief and plausibility measures and a
  mass function.
  The conditional belief and plausibility of $A$ given $B$ are:
  \begin{align}
    \label{eqn:4:ConditionalBelief}
    \bel(A \mid B)
     & = \sum_{C : C \subseteq A} m \oplus m_B(C)
    \\[2ex]
    \label{eqn:4:ConditionalPlausibility}
    \pl(A \mid B)
     & = \sum_{C : C \cap A \neq \emptyset} m \oplus m_B(C)
  \end{align}
  where $m_B(B) = 1$, $m_B(C) = 0\ \forall\ C \neq B$, and $\pl(B) > 0$.
\end{dfn}

\begin{thm}[Relation to plausibility]
  \label{thm:4:RelationToPlausibility}
  Let $W$ be a set of possible worlds, $A, B \subseteq W$ be propositions, and
  $\pl : 2^W \to [0, 1]$ be a plausibility measure.
  \begin{equation}
    \label{eqn:4:RelationToPlausibility}
    \pl(A \mid B) = \frac{\pl(A \cap B)}{\pl(B)}
  \end{equation}
  \begin{proof}
    By definitions~\ref{dfn:4:ConditionalBeliefPlausibility} and
    \ref{dfn:4:DempstersRuleOfCombination} and $\sum_D m(D) = 1$:
    \begin{align*}
      \pl(A \mid B)
       & = \sum_{C : C \cap A \neq \emptyset} m \oplus m_B(C)
      = \sum_{C : C \cap A \neq \emptyset} \frac{\sum_{D : D \cap B = C} m(D)}{1 - \sum_{D : D \cap B = \emptyset} m(D)}
      \\[3ex]
       & = \frac{\sum_{C : C \cap A \neq \emptyset} \sum_{D : D \cap B = C} m(D)}{1 - \sum_{D : D \cap B = \emptyset} m(D)}
      =  \frac{\sum_{C : C \cap A \neq \emptyset} \sum_{D : D \cap B = C} m(D)}{\sum_{D : D \cap B \neq \emptyset} m(D)}
      \\[3ex]
       & = \frac{\sum_{D : (D \cap B) \cap A \neq \emptyset} m(D)}{\sum_{D : D \cap B \neq \emptyset} m(D)}
      = \frac{\sum_{D : D \cap (B \cap A) \neq \emptyset} m(D)}{\sum_{D : D \cap B \neq \emptyset} m(D)}
      = \frac{\pl(A \cap B)}{\pl(B)}
    \end{align*}
  \end{proof}
\end{thm}

\begin{thm}[Relation to belief]
  Let $W$ be a set of possible worlds, $A, B \subseteq W$ be propositions, and
  $\bel : 2^W \to [0, 1]$ be a belief measure.
  \begin{equation}
    \label{eqn:4:RelationToBelief}
    \bel(A \mid B) = \frac{\bel(A \cup B^c) - \bel(B^c)}{1 - \bel(B^c)}
  \end{equation}
  \begin{proof}
    By theorems~\ref{thm:4:DualRelationship} and
    \ref{thm:4:RelationToPlausibility}:
    \begin{align*}
      \bel(A \mid B)
       & = 1 - \pl(A^c \mid B)
      = 1 - \frac{\pl(A^c \cap B)}{\pl(B)}
      = \frac{\pl(B) - \pl(A^c \cap B)}{\pl(B)}
      \\[3ex]
       & = \frac{1 - \bel(B^c) - (1 - \bel((A^c \cap B)^c))}{1 - \bel(B^c)}
      = \frac{\bel(A \cup B^c) - \bel(B^c)}{1 - \bel(B^c)}
    \end{align*}
  \end{proof}
\end{thm}

Definition~\ref{dfn:4:DempstersRuleOfCombination} produces counterintuitive
results if the mass functions are highly inconsistent, i.e., the denominator $1
  - \sum_{(B, C) : B \cap C \neq \emptyset} m_1(B) m_2(C) \approx 0$.
Alternative combination rules have been proposed to address this problem.
