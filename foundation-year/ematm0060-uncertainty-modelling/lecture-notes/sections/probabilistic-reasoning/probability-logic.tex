
\subsection{Probability logic}

Probability theory is not truth-functional.
Knowledge of $P(A)$ and $P(B)$ does not imply knowledge of, e.g., $P(A \cap
  B)$, but it does imply upper and lower bounds
(theorems~\ref{thm:3:BoundsIntersection} and \ref{thm:3:BoundsUnion}).

\begin{thm}[Intersection bounds]
  \label{thm:3:BoundsIntersection}
  Let $A, B \subseteq W$.
  \begin{equation}
    \label{eqn:3:BoundsIntersection}
    \max(0, P(A) + P(B) - 1) \leq P(A \cap B) \leq \min(P(A), P(B))
  \end{equation}
  \begin{proof}
    By definition~\ref{def:2:ProbabilityMeasure}:
    \begin{align}
      P(A) & = P(A \cap B) + P(A \cap B^c) \label{eqn:3:BoundsIntersection1}
      \\
      P(B) & = P(A \cap B) + P(A^c \cap B) \label{eqn:3:BoundsIntersection2}
    \end{align}
    Since $P(A \cap B^c) \geq 0$ and $P(A^c \cap B) \geq 0$:
    \begin{equation}
      P(A \cap B) \leq \min(P(A), P(B)) \label{eqn:3:BoundsIntersection3}
    \end{equation}
    By theorem~\ref{thm:2:GeneralAdditivity}:
    \begin{equation}
      P(A \cap B) = P(A) + P(B) - P(A \cup B) \label{eqn:3:BoundsIntersection4}
    \end{equation}
    Since $P(A \cup B) \leq 1$ and $P(A \cap B) \geq 0$:
    \begin{equation}
      P(A \cap B) \geq \max(0, P(A) + P(B) - 1) \label{eqn:3:BoundsIntersection5}
    \end{equation}
  \end{proof}
\end{thm}

\begin{thm}[Union bounds]
  \label{thm:3:BoundsUnion}
  Let $A, B \subseteq W$.
  \begin{equation}
    \max(P(A), P(B)) \leq P(A \cup B) \leq \min(P(A) + P(B), 1)
  \end{equation}
  \begin{proof}
    By theorems~\ref{thm:2:GeneralAdditivity} and \ref{thm:3:BoundsIntersection}:
    \begin{equation}
      P(A \cup B) \geq P(A) + P(B) - \min(P(A), P(B)) = \max(P(A), P(B))
      \label{eqn:3:BoundsUnion1}
    \end{equation}
    By theorem~\ref{thm:3:BoundsIntersection}:
    \begin{align}
      P(A \cup B) & \leq P(A) + P(B) - \max(0, P(A) + P(B) - 1)
      \\
                  & \leq \min(P(A) +
      P(B), 1) \label{eqn:3:BoundsUnion3}
    \end{align}
  \end{proof}
\end{thm}

Theorems~\ref{thm:2:Complement},
\ref{thm:3:BoundsIntersection}, and \ref{thm:3:BoundsUnion} define a
truth-functional logic for probability intervals.
For a proposition $A \subseteq W$, it can be used to infer upper and lower
bounds on $P(A)$, i.e., $P(A) \in [L(A),\ U(A)]$.

\begin{thm}[Complement rule]
  \begin{equation}
    P(A^c) \in [1 - U(A),\ 1 - L(A)]
  \end{equation}
\end{thm}

\begin{thm}[Intersection rule]
  \begin{equation}
    P(A \cap B) \in [\max(0, L(A) + L(B) - 1),\ \min(U(A), U(B))]
  \end{equation}
\end{thm}

\begin{thm}[Union rule]
  \begin{equation}
    P(A \cup B) \in [\max(L(A), L(B)),\ \min(U(A) + U(B),\ 1)]
  \end{equation}
\end{thm}

\begin{thm}[Conditional rule]
  \begin{equation}
    P(A) \in [L(A \mid B) L(B),\ (U(A \mid B) - 1) U(B) + 1]
  \end{equation}
  \begin{proof}
    By theorem~\ref{thm:2:TotalProbability} and
    definition~\ref{def:2:ProbabilityMeasure}, i.e., $P(A \mid B^c) \in [0, 1]$:
    \begin{itemize}
      \item If $P(A \mid B^c) = 0$, then $P(A) = P(A \mid B) P(B)$ (lower bound).
      \item If $P(A \mid B^c) = 1$, then $P(A) = (P(A \mid B) - 1) P(B) + 1$ (upper bound).
    \end{itemize}
  \end{proof}
\end{thm}

\begin{thm}[Jeffrey's rule]
  \begin{equation}
    P^\prime(A) \in [L(A \mid B) L(B),\ (U(A \mid B) - 1) U(B) + 1]
  \end{equation}
\end{thm}

The upper and lower bounds of this logic are
wider than probability theory constrains them to be.
This is because the logic does not account for logical dependencies between
propositions and their relations to definition~\ref{def:2:ProbabilityMeasure}.
