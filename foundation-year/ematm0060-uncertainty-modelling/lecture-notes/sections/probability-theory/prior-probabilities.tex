\subsection{Prior probabilities}

Bayesian reasoning is difficult without prior knowledge of the hypotheses.
The most common approach to this problem is \textit{Laplace's principle of insufficient reasoning}:
\begin{displayquote}
  In the absence of any other information, all hypotheses under consideration
  should be assumed to be equally probable, i.e., the probability distribution
  should be \textit{uniform}.
\end{displayquote}
The uniform distribution is the least informative
(theorem~\ref{thm:3:MaximumEntropyDistribution}).

Probability theory conflates uncertainty (a lack of knowledge as to the true
possible world, quantified by a probability measure) and ignorance (a lack of
knowledge that makes it difficult to quantify one's beliefs).
A different approach is to use a theory of uncertainty that differentiates
between uncertainty and ignorance, e.g., Dempster-Shafer theory.
