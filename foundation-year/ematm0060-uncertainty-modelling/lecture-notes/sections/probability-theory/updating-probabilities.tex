\subsection{Updating probabilities and Bayesian reasoning}

If an agent learns that a proposition $A$ is true, then it should update its
probability distribution to reflect that $w^* \in A$.
If an agent learns that a proposition $A$ is false, then it cannot update its
probability distribution.

\begin{dfn}[Conditional probability]
  \label{def:2:ConditionalProbability}

  Let $P$ be a probability measure such that $P(B)>0$.
  The conditional probability of $A$ given $B$ is:
  \begin{equation}
    \label{eqn:2:ConditionalProbability}
    P(A \mid B) = \frac{P(A \cap B)}{P(B)}
  \end{equation}
\end{dfn}

Let $H = \{H_i \mid i \in 1 .. k\}\subseteq W$ be a set of
hypotheses.
$H$ are:
\begin{itemize}
  \item \textit{mutually exclusive} if $H_i \cap H_j = \emptyset\ \forall\ i \neq j$; and
  \item \textit{exhaustive} if $\bigcup_{i = 1}^{k}
          H_i = W$.
\end{itemize}
In the context of Bayes' theorem:
\begin{itemize}
  \item $P(H_i \mid E)$ are the \textit{posterior} probabilities;
  \item $P(H_i)$ are the \textit{prior} probabilities; and
  \item $P(E \mid H_i)$ are the \textit{likelihoods}.
\end{itemize}

It is generally impractical to evaluate the posterior probabilities.
But it is more practical to estimate the prior probabilities and the
likelihoods.
Bayes' theorem can be used to estimate the posterior probabilities.

\begin{thm}[Bayes' theorem]
  \label{thm:2:BayesTheorem}

  Let $\{H_i \mid i \in 1 .. k\}\subseteq W$ such that $H_i \cap H_j = \emptyset\
    \forall\ i \neq j$ and $\bigcup_{i = 1}^{k} H_i = W$.
  Then, for $E \subseteq W$ such that $P(E) > 0$:
  \begin{equation}
    \label{eqn:2:BayesTheorem}
    P(H_i \mid E)
    = \frac{P(E \mid H_i)P(H_i)}{\sum_{j = 1}^{k} P(E \mid H_j) P(H_j)}\
    \forall\ i \in 1 .. k
  \end{equation}
  \begin{proof}
    By definition \ref{def:2:ConditionalProbability}:
    \begin{equation}
      \label{eqn:2:BayesTheorem1}
      P(E \mid H_i) = \frac{P(H_i \cap E)}{P(H_i)}\ ,\
      P(H_i \mid E) = \frac{P(E \cap H_i)}{P(E)}
    \end{equation}
    Therefore:
    \begin{equation}
      \label{eqn:2:BayesTheorem2}
      P(H_i \mid E) = \frac{P(E \mid H_i) P(H_i)}{P(E)}
    \end{equation}
    $\{H_i\mid i\in 1 .. k\}$ is a partition of $W$, hence:
    \begin{equation}
      \label{eqn:2:BayesTheorem3}
      \sum_{i = 1}^{k}P(H_i \mid E) = 1
    \end{equation}
    Substituting \ref{eqn:2:BayesTheorem3} into \ref{eqn:2:BayesTheorem2} gives:
    \begin{equation}
      \label{eqn:2:BayesTheorem5}
      1 = \sum_{i = 1}^{k}\frac{P(E \mid H_i)P(H_i)}{P(E)}
      = \frac{\sum_{i = 1}^{k}
        P(E \mid H_i) P(H_i)}{P(E)}
    \end{equation}
    Substituting
    \ref{eqn:2:BayesTheorem5} into \ref{eqn:2:BayesTheorem3} gives
    \ref{eqn:2:BayesTheorem}.
  \end{proof}
\end{thm}

Conditional inference applies a general rule to a specific instance:
\begin{itemize}
  \item $P(A \mid B)$ is a conditional probability defined on a set of instances; and
  \item $P(B)$ is a probability defined on a specific instance.
\end{itemize}

In Bayes' theorem, proposition $E$ is certainly true, i.e., $w^* \in E$.
Theorem~\ref{thm:2:JeffreysRule} is a generalisation of Bayes' theorem where
$P(E) \in [0, 1]$.
Its result is a new probability measure $P^\prime$.

\begin{thm}[Jeffrey's rule]
  \label{thm:2:JeffreysRule}
  \begin{equation}
    \label{eqn:2:JeffreysRule}
    P^\prime(H) = P(H \mid E) P^\prime(E) + P(H \mid E^\prime) (1 - P^\prime(E))
  \end{equation}
\end{thm}

Bayesian reasoning strongly
depends on the prior probabilities.
Any posterior probability can be obtained by selecting an appropriate prior
probability.
I.e., the likelihoods alone do not determine the posterior probabilities.

\begin{dfn}[Bayes factor]
  \label{def:2:BayesFactor}

  Given evidence $E$ and a hypothesis $H$, the Bayes factor is:
  \begin{equation}
    \label{eqn:2:BayesFactor}
    F = \frac{P(E \mid H)}{P(E \mid H^c)}
  \end{equation}
\end{dfn}
