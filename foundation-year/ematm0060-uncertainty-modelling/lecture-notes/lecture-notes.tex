% !TEX root=lecture-notes.tex

\documentclass[a4paper]{extarticle}
\usepackage{amsmath}
\usepackage{amssymb}
\usepackage{amsthm}
\usepackage{csquotes}
\usepackage{dsfont}
\usepackage{enumerate}
\usepackage{hyperref}
\usepackage{import}
\usepackage{microtype}
\usepackage{thmtools}

\numberwithin{equation}{section}
\theoremstyle{plain}
\newtheorem{thm}{Theorem}[subsection]
\newtheorem{dfn}[thm]{Definition}

\DeclareMathOperator*{\bel}{bel}
\DeclareMathOperator*{\pl}{pl}
\newcommand{\card}[1]{|#1|}

\title{Uncertainty Modelling for Intelligent Systems}
\author{Tim Lawson}
\date{\today}

\begin{document}
\maketitle

These notes are based on the lecture notes for the unit
\href{https://www.bris.ac.uk/unit-programme-catalogue/UnitDetails.jsa?
  ayrCode=23\%2F24&unitCode=EMATM1120}
{Uncertainty Modelling for Intelligent Systems in 2023/24} at the University of
Bristol.

\tableofcontents

\newpage
\listoftheorems[numwidth=3em,title=Theorems]

\import{sections/}{possible-worlds.tex}
\import{sections/}{probability-theory.tex}
\import{sections/}{probabilistic-reasoning.tex}
\import{sections/}{ignorance-uncertainty.tex}

\end{document}
