\documentclass[a4paper]{extarticle}
\usepackage{../../latex/packages/tslwn-preamble}

\newcommand{\tpr}{t^\prime}
\newcommand{\dtpr}{\text{d}t^\prime}
\newcommand{\ihbar}{\frac{i}{\hbar}}
\newcommand{\propagator}{K(\vec{r}_f, \vec{r}_0, t)}

\title{Advanced Quantum Theory}
\author{Tim Lawson}
\date{\today}

\begin{document}
\maketitle

\tableofcontents

\section{intro}

\begin{itemize}
  \item Identity for discrete and eigenmomentum states
  \item Almost ready to do path integrals
  \item Solution of time evolution of wavefunction works if you ignore that H is an operator
\end{itemize}

If you're more careful, you'd want to consider that it's an operator.
What's the square of an operator? Apply it twice.
Conclusion is that there's a nice way to study time-evolution in QM with a time-evolution operator.

Next, we can define a propagator.

The propagator is referred to as the matrix element of the time-evolution operator.
For a discrete basis, it produces a matrix.

\begin{dfn}
  [Time-evolution operator]
  \begin{equation}
    e^{-\frac{i}{\hbar} \hat{H} t}
  \end{equation}
\end{dfn}

\begin{dfn}
  [Propagator]
  \begin{equation}
    K(\vec{r}_f, \vec{r}_0, t) = \bra{\vec{r}_f} e^{-i \hat{H} t / \hbar} \ket{\vec{r}_0}
  \end{equation}
\end{dfn}

At $t = 0$, you have the initial condition which is the delta function localised
at $\vec{r}_0$, i.e., it's zero everywhere except at that position.

The boundary conditions are less clear in this definition (than?).
Mathematically implicit in terms of the space to which the Hamiltonian is applied.

\paragraph{Path integral}

\begin{equation}
  K(\vec{r}_f, \vec{r}_0, t) = \int D \left[\vec{r}\right] e^{\frac{i}{\hbar} S\left[\vec{r}\right] }
\end{equation}
where the integral is over trajectories $\vec{r}_0$ to $\vec{r}_f$ in time $t$.
Crucially, the integral is over trajectories that do not necessarily satisfy the
classical laws of motion.

This raises lots of issues.
Mathematically, how can we integrate over trajectories?
They are described by functions $\vec{r}(\tpr)$ where $\vec{r}(0) = \vec{r}_0$ and
$\vec{r}(\tpr) = \vec{r}_f$ ($t$ is reserved for the duration of the trajectory).
Problems will be solved by the derivation.
Discretisation into timesteps.
The function values after each timestep can be turned into a vector.
Integrals over vectors we can do, then you need to make it into a limit.
Somehow, discretising is the answer to how we integrate over trajectories (functions).

Next we'll talk about the integrand: the action $S$.
\begin{equation}
  S = \int_0^t L\left(\vec{r}(t), \dot{\vec{r}}(t)\right) \dtpr
\end{equation}
The square brackets are a convention when dealing with a function (functionals).

It's exact but it's only exact because it includes trajectories that don't satisfy
the classical laws of motion.
The Hamiltonian appears in classical mechanics but $i$ and $\hbar$ only in QM.

Eventually, we can understand the relation between the two.
Lagrangian mechanics.
Should not be changed to linear order (small changes) in trajector.
When stationary, some trajectories dominate and those are precisely those that
satsify the classical laws of motion.
Lagrange equations are equivalent to the action being stationary.

Derivation: start with short times and generalise to arbitrary times.

\paragraph{Short times}

\begin{align}
  \hat{H}
   & = \frac{\vec{\hat{p}}^2}{2m} + \hat{U}(\vec{\hat{r}})                            \\[1.5ex]
  \propagator
   & = \bra{\vec{r}_f} e^{-\ihbar (\hat{T} + \hat{U}) t} \ket{\vec{r}_0}              \\[1.5ex]
   & \approx \bra{\vec{r}_f} -\ihbar (\hat{T} + \hat{U}) t \ket{\vec{r}_0}            \\[1.5ex]
   & \approx \bra{\vec{r}_f} (1 - \ihbar\hat{T})(1 - \ihbar\hat{U}) t \ket{\vec{r}_0} \\[1.5ex]
   & \approx \bra{\vec{r}_f} e^{-\ihbar\hat{T}t} e^{-\ihbar\hat{U}t} \ket{\vec{r}_0}
\end{align}
The extra term is OK because we're working to linear order so can discard the
quadratic term.
Now we can apply the momentum operator to the wavefunction.
\begin{align}
   & = \bra{\vec{r}_f} e^{-\ihbar\hat{T}t} e^{-\ihbar U(\vec{r}_0) t} \ket{\vec{r}_0} \\
   & = \bra{\vec{r}_f} e^{-\ihbar\hat{T}t} \ket{\vec{r}_0} e^{-\ihbar U(\vec{r}_0) t}
\end{align}
Call the first bit $A$.
Density, resolution of identity.
$\hat{T}$ is important because it involves momentum.
Hard to combine momentum with position eigenstate so could involve a momentum eigenstate.
\begin{align}
  A
   & = \bra{\vec{r}_f} e^{-\ihbar\hat{T}t} \int \text{d}^n \ket{\vec{p}} \braket{\vec{p}}{\vec{r}_0}                  \\
   & = \int \text{d}^n\vec{p} \bra{\hat{\vec{r}_f}} e^{-\ihbar\frac{\hat{\vec{p}}}{2m^2}} \braket{\vec{p}}{\vec{r}_0} \\
   & = \int \text{d}p e^{-\ihbar\frac{p^2}{2m}t} \braket{\vec{r}_f}{\vec{p}} \braket{\vec{p}}{\vec{r}_0}
\end{align}
Momentunm operator will give the eigenvalue.
Set a single p-hat, apply for p, etc., get a number.
Can rewrite the brakets $\braket{\vec{r}_f}{\vec{p}} = \omega(\vec{p})$.

Complex conjugate so can exchange two factors if we conjugate.


Position operator applied to the wavefunction is just multiplying with its argument.
But generally it's an operator so we keep the hat.
Split into kinetic and potential terms.

\dots

What's useful? We have a position, so applying the position operator to it is useful.
There might be ways to apply an operator to the bra but not sure.
Linear term in exponential for small t.
Small t means linear order.
Try to factorise it in different ways.

\end{document}