\documentclass[a4paper]{extarticle}
\usepackage{../../latex/packages/tslwn-preamble}

\newcommand{\tpr}{t^\prime}
\newcommand{\dtpr}{\text{d}t^\prime}
\newcommand{\ddtpr}{\frac{\text{d}}{\text{d}t^\prime}}
\newcommand{\ihbar}{\frac{i}{\hbar}}
\newcommand{\propagator}{K(\vec{r}_f, \vec{r}_0, t)}

\title{Advanced Quantum Theory}
\author{Tim Lawson}
\date{\today}

\begin{document}
\maketitle

\tableofcontents
\listoftheorems

\section{Basics}

\subsection{Classical mechanics}

\subsection{Quantum mechanics}

In quantum mechanics, variables $q_\alpha, p_\alpha$ are replaced by operators
$\hat{q}_\alpha, \hat{p}_\alpha$ that act on wavefunctions.
In the position representation, the operator $\hat{q}_\alpha$ acts by
multiplying the wavefunction by $q_\alpha$.

\subsubsection{The delta `function'}

\section{intro}

\begin{itemize}
  \item Identity for discrete and eigenmomentum states
  \item Almost ready to do path integrals
  \item Solution of time evolution of wavefunction works if you ignore that H is an operator
\end{itemize}

If you're more careful, you'd want to consider that it's an operator.
What's the square of an operator? Apply it twice.
Conclusion is that there's a nice way to study time-evolution in QM with a time-evolution operator.

Next, we can define a propagator.

The propagator is referred to as the matrix element of the time-evolution operator.
For a discrete basis, it produces a matrix.

\begin{dfn}
  [Time-evolution operator]
  \begin{equation}
    e^{-\frac{i}{\hbar} \hat{H} t}
  \end{equation}
\end{dfn}

\begin{dfn}
  [Propagator]
  \begin{equation}
    K(\vec{r}_f, \vec{r}_0, t) = \bra{\vec{r}_f} e^{-i \hat{H} t / \hbar} \ket{\vec{r}_0}
  \end{equation}
\end{dfn}

At $t = 0$, you have the initial condition which is the delta function localised
at $\vec{r}_0$, i.e., it's zero everywhere except at that position.

The boundary conditions are less clear in this definition (than?).
Mathematically implicit in terms of the space to which the Hamiltonian is applied.

\paragraph{Path integral}

\begin{equation}
  K(\vec{r}_f, \vec{r}_0, t) = \int D \left[\vec{r}\right] e^{\frac{i}{\hbar} S\left[\vec{r}\right] }
\end{equation}
where the integral is over trajectories $\vec{r}_0$ to $\vec{r}_f$ in time $t$.
Crucially, the integral is over trajectories that do not necessarily satisfy the
classical laws of motion.

This raises lots of issues.
Mathematically, how can we integrate over trajectories?
They are described by functions $\vec{r}(\tpr)$ where $\vec{r}(0) = \vec{r}_0$ and
$\vec{r}(\tpr) = \vec{r}_f$ ($t$ is reserved for the duration of the trajectory).
Problems will be solved by the derivation.
Discretisation into timesteps.
The function values after each timestep can be turned into a vector.
Integrals over vectors we can do, then you need to make it into a limit.
Somehow, discretising is the answer to how we integrate over trajectories (functions).

Next we'll talk about the integrand: the action $S$.
\begin{equation}
  S = \int_0^t L\left(\vec{r}(t), \dot{\vec{r}}(t)\right) \dtpr
\end{equation}
The square brackets are a convention when dealing with a function (functionals).

It's exact but it's only exact because it includes trajectories that don't satisfy
the classical laws of motion.
The Hamiltonian appears in classical mechanics but $i$ and $\hbar$ only in QM.

Eventually, we can understand the relation between the two.
Lagrangian mechanics.
Should not be changed to linear order (small changes) in trajector.
When stationary, some trajectories dominate and those are precisely those that
satsify the classical laws of motion.
Lagrange equations are equivalent to the action being stationary.

Derivation: start with short times and generalise to arbitrary times.

\paragraph{Short times}

\begin{align}
  \hat{H}
   & = \frac{\vec{\hat{p}}^2}{2m} + \hat{U}(\vec{\hat{r}})                            \\[1.5ex]
  \propagator
   & = \bra{\vec{r}_f} e^{-\ihbar (\hat{T} + \hat{U}) t} \ket{\vec{r}_0}              \\[1.5ex]
   & \approx \bra{\vec{r}_f} -\ihbar (\hat{T} + \hat{U}) t \ket{\vec{r}_0}            \\[1.5ex]
   & \approx \bra{\vec{r}_f} (1 - \ihbar\hat{T})(1 - \ihbar\hat{U}) t \ket{\vec{r}_0} \\[1.5ex]
   & \approx \bra{\vec{r}_f} e^{-\ihbar\hat{T}t} e^{-\ihbar\hat{U}t} \ket{\vec{r}_0}
\end{align}
The extra term is OK because we're working to linear order so can discard the
quadratic term.
Now we can apply the momentum operator to the wavefunction.
\begin{align}
   & = \bra{\vec{r}_f} e^{-\ihbar\hat{T}t} e^{-\ihbar U(\vec{r}_0) t} \ket{\vec{r}_0} \\
   & = \bra{\vec{r}_f} e^{-\ihbar\hat{T}t} \ket{\vec{r}_0} e^{-\ihbar U(\vec{r}_0) t}
\end{align}
Call the first bit $A$.
Density, resolution of identity.
$\hat{T}$ is important because it involves momentum.
Hard to combine momentum with position eigenstate so could involve a momentum eigenstate.
\begin{align}
  A
   & = \bra{\vec{r}_f} e^{-\ihbar\hat{T}t} \int \text{d}^n \ket{\vec{p}} \braket{\vec{p}}{\vec{r}_0}                  \\
   & = \int \text{d}^n\vec{p} \bra{\hat{\vec{r}_f}} e^{-\ihbar\frac{\hat{\vec{p}}}{2m^2}} \braket{\vec{p}}{\vec{r}_0} \\
   & = \int \text{d}p e^{-\ihbar\frac{p^2}{2m}t} \braket{\vec{r}_f}{\vec{p}} \braket{\vec{p}}{\vec{r}_0}
\end{align}
Momentunm operator will give the eigenvalue.
Set a single p-hat, apply for p, etc., get a number.
Can rewrite the brakets $\braket{\vec{r}_f}{\vec{p}} = \omega(\vec{p})$.

Complex conjugate so can exchange two factors if we conjugate.

Position operator applied to the wavefunction is just multiplying with its argument.
But generally it's an operator so we keep the hat.
Split into kinetic and potential terms.

\dots

What's useful? We have a position, so applying the position operator to it is useful.
There might be ways to apply an operator to the bra but not sure.
Linear term in exponential for small t.
Small t means linear order.
Try to factorise it in different ways.

\begin{align*}
  A = \prod_{k = 1}^{n} = \frac{1}{2\pi} \int_{}
  \exp\left[
    -\ihbar \frac{1}{2m}
    \right]
\end{align*}

Use the shifted Fresnel integral to evaluate $A$ above.

\begin{align*}
  A = \prod_{k = 1}^{n} \left( \frac{m}{2\pi i\hbar t} \right)^{\frac{1}{2}}
  \exp\left[ \ihbar \frac{m}{2} \left( \frac{r_{f, k} - r_{0, k}}{t} \right)^2 t \right]
\end{align*}

General case (arbitrary times): expect an integral.
We have an exponential so it'd be nice if it had something to do with the
exponential above.

Difference between positions over time is a velocity.
Multiplied by $m/2$ gives a kinetic energy.
Then minus the potential energy gives the action.
And multiplied by $t$.

For small $t$ and $\vec{r}_f$ close to $\vec{r}_0$, the dominant path between
them is a straight line.
Then we get exactly the kinetic energy, etc.

$\frac{1}{2}m \left( \frac{r_{f, k} - r_{0, k}}{t} \right)^2$ is a good
approximation for the kinetic energy for small $t$.
$U(\vec{r}_0)$ is a good approximation for the potential energy for small $t$.
Therefore, the difference between the two is a good approximation for the
Lagrangian along this trajectory.
Integrating over something that's approximately constant is approximated by
multiplying the $\approx$ constant by $t$.

\paragraph{2.2 Arbitrary times}

If we want to use the propagator for short times to derive the propagator for
arbitrary times, then we want to divide the integral into intervals that are
short.

Divide $t$ into intervals $\tau = \frac{t}{N}, N \to \infty$.
$N$ is the number of timesteps whereas $n$ is the dimension.

\begin{align*}
  \propagator
   & = \bra{\vec{r}_f} e^{-\ihbar \hat{H} t} \ket{\vec{r}_0}                     \\
   & = \bra{\vec{r}_f} \left( e^{-\ihbar \hat{H} \tau} \right)^N \ket{\vec{r}_0}
\end{align*}

We can insert $N - 1$ resolutions of identity
$1 = \int \text{d}^n r_{i} \ket{\vec{r}_{i}} \bra{\vec{r}_{i}} : i \in 1 .. N - 1$.
Let $\vec{r}_N := \vec{r}_f$.
Hence:
\begin{align*}
   & = \int \text{d}^n r_1 \dots \text{d}^n r_{N - 1}
  \bra{\vec{r}_N} e^{-\ihbar \hat{H} \tau} \ket{\vec{r}_{N - 1}} \dots
  \bra{\vec{r}_1} e^{-\ihbar \hat{H} \tau} \ket{\vec{r}_{0}}                      \\
   & = \int \text{d}^n r_1 \dots \text{d}^n r_{N - 1} \prod_{j = 0}^{N - 1}
  \bra{\vec{r}_{j + 1}} e^{-\ihbar \hat{H} \tau} \ket{\vec{r}_{j}}                \\
   & \approx \int \text{d}^n r_1 \dots \text{d}^n r_{N - 1} \prod_{j = 0}^{N - 1}
  \left(\frac{m}{2\pi i\hbar\tau}\right)^{\frac{n}{2}}
  \exp\left[ \ihbar\left[
      \frac{1}{2}m\left( \frac{\vec{r}_{j + 1} - \vec{r}_{j}}{\tau} \right)^2 -
      U(\vec{r}_j)
  \right] \tau \right]                                                            \\
   & \approx
  \left(\frac{m N}{2\pi i\hbar t}\right)^{\frac{n N}{2}}
  \int \text{d}^n r_1 \dots \text{d}^n r_{N - 1}
  \exp\left[
    \ihbar\sum_{j = 0}^{N - 1} \left[
      \frac{1}{2}m\left( \frac{\vec{r}_{j + 1} - \vec{r}_{j}}{\tau} \right)^2 -
      U(\vec{r}_j)
      \right] \tau
    \right]
\end{align*}
Note that we have substituted $\tau = \frac{t}{N}$ in the prefactor.

Interpret the integration variables $\vec{r}_j$ as positions at the intermediate
times $t_j = j \tau$.
The initial and final positions $\vec{r}_0$ and $\vec{r}_N$ are given.
The integral is over all possible trajectories $\vec{r}_1, \dots, \vec{r}_{N - 1}$.

In the limit $N \to \infty$, the sum becomes an integral.
This interpretation applies in the case that the paths are smooth.
Next time, we'll talk about the case in which trajectories are not differentiable.

The exponent is close to an action.
Integrating over positions at discretised points along trajectories.

\begin{itemize}
  \item $(\vec{r}_{j + 1} - \vec{r}_j)/\tau$ is a good approximation for the velocity;
  \item the summand is a good approximation for the Lagrangian between $j\tau$ and $(j + 1)\tau$;
  \item the sum is a good approximation for the action (multiplication by $t$)
\end{itemize}

The approximation becomes better as we discretise more finely.
In the limit $N \to \infty$, the approximation becomes exact, i.e., we can
replace $=$ by $\approx$ and the sum by an integral.

The general definition of the integral is the sum of the contributions of
the little intervals, i.e., their values multiplied by their lengths.

\begin{equation}
  \lim_{N \to \infty} \sum_{j = 0}^{N - 1} \left[
    \frac{1}{2}m\left( \frac{\vec{r}_{j + 1} - \vec{r}_{j}}{\tau} \right)^2 -
    U(\vec{r}_j)
    \right] \tau
  = \int_0^t L(\vec{r}(\tpr), \dot{\vec{r}}(\tpr)) \dtpr
\end{equation}

$D\left[\vec{r}\right]$ is some kind of integration measure.
We define it as the limit $N \to \infty$...

Result:
\begin{equation}
  \int D\left[\vec{r}\right] \cdots
  \lim_{N \to \infty} prefactor \int d^n r_1 \dots d^n r_{N - 1} \cdots
\end{equation}
The dots are any property of the trajectory of the path.

Thus:
\begin{equation}
  \propagator = \int_{\vec{r}(0) = \vec{r}_0, \vec{r}(t) = \vec{r}_f}
  D\left[\vec{r}\right] e^{\ihbar S\left[\vec{r}\right]}
\end{equation}

Remarks

A pure mathematician (analyst) would be more careful/rigorous.
We've assumed that the trajectories are smooth.
The velocity only works if the trajectory is differentiable.

Rigorous treatment requires care with non-differentiable trajectories.
Integration measure.
Fewer issues with discretised version because you don't have to work with
derivatives.

The prefactor looks dangerous because it includes $N$.
It has to be seen in the context of the integral.
You have to check that it converges for a given integrand.

Condition is that all the possible points are accessible for r.
If some positions are impossible, can get additional phase.

Hamiltonian mechanics version
The action is first defined as the integral over the Lagrangian.
Probably don't want exactly that, something with the Hamiltonian.

In this context, $H = \vec{p}\cdot\vec{r} - L$.
Example of a Legendre transformation.
We have to be careful that we're writing the Hamiltonian in the same way.

\begin{equation}
  \propagator
  = \int D\left[\vec{r}\right] D\left[vec{p}\right]
  \exp\left(
  \ihbar \int \dtpr \left[
      \vec{p}(\tpr) \cdot \dot{\vec{r}}(\tpr) - H(\vec{r}(\tpr), \vec{p}(\tpr))
      \right]
  \right)
\end{equation}
where the integral is over trajectories in phase space.

Homework: 2.1.

Where to get momentum integrals? Modelling on previous lectures.
Can use resolution of identity in phase space.
Don't have to change where I'm doing that because I already did to split the
short-time propagator. Could just keep p-integral.

How many variables?
$\vec{r}_0$ and $\vec{r}_f$ are fixed.
Integrate over $\vec{r}_1, \dots, \vec{r}_{N - 1}$.
How many momentum variables are integrated over when discretised? $N$.
You have $N$ short-time propagators, each of which involves a $p$ integral.
Be careful with indices!
There isn't complete symmetry because you have the initial and final position,
whereas you don't have the initial and final momentum.

Homework 2.6a.

\begin{equation}
  \bra{p_f} e^{-\ihbar \hat{H} t} \ket{p_0},
  \hat{H} = \frac{\hat{p}^2}{2} + \hat{U}(\vec{x})
\end{equation}
for short times.
\begin{equation}
  \bra{p_f} e^{-\ihbar (\hat{T} + \hat{U}) t} \ket{p_0}
\end{equation}
Separate exponentials (for short times).
Now we have p on the right instead of r, so we want the T operator on the right,
which involves the momentum.
\begin{equation}
  \bra{p_f} e^{-\ihbar \hat{U} t} e^{-\ihbar \hat{T} t} \ket{p_0}
\end{equation}
Could try to apply T to the state on the left with a dagger.
Momentum operator plus momentum eigenstate.
Insert resolution of identity for position.
U has x has an operator inside, so applying it to the ket gives a position eigenstate.
Then we have lots of numbers (good).
\begin{equation}
  \int \dx \exp(-\ihbar ( \frac{p^2}{2} + U(x)) t) \braket{p_f}{x} \braket{x}{p_0}
\end{equation}
Replace brakets with states.
\begin{equation}
  \frac{1}{2\pi\hbar} \intfty\dx
  \exp(-\ihbar ( \frac{p^2}{2} + U(x)) t) \exp(-\frac{i}{\hbar} (p_f - p_0) x)
\end{equation}

Similarly with $t = N\tau$, insert lots of resolutions of identity.

\section{Tue 14 Nov}

Remarks:

``Classical trajectories'': trajectories that satisfy the classical laws of motion.
Hamilton's principle: trajectories that satsify the classical laws of motion
are stationary (how?).

\begin{itemize}
  \item all trajectories integrated over; expect classical trajectories to dominate
  \item the action is stationary for classical trajectories
  \item trajectories close to a classical trajectory have similar actions and
        make similar contributions to the path integral and hence an important
        joint contribution - that means they dominate the integral
\end{itemize}

When is it a good approximation? We used the fact that $\hbar$ is small in the
argument that classical trajectories dominate. Really we mean that it's small
with respect to the dimensions of what we're evaluating, e.g., classical
actions (distances, etc.).

\section*{Appendix}

\begin{thm}
  [Gauss integral]
  \label{thm:a:gauss-integral}
  Let $a \in \mathbb{R}$ and $a > 0$.
  \begin{equation}
    \label{eqn:a:gauss-integral}
    \intfty e^{-ax^2} \dx = \sqrt{\frac{\pi}{a}}
  \end{equation}
\end{thm}

\begin{thm}
  [Fresnel integral]
  \label{thm:a:fresnel-integral}
  Let $a \in \mathbb{R}$.
  \begin{equation}
    \label{eqn:a:fresnel-integral}
    \intfty e^{\mp iax^2}
    = \sqrt{\frac{\pi}{\abs{a}}} \exp\left(\pm i\frac{\pi}{4} \emph{sgn}\,a\right)
    = \sqrt{\frac{\pi}{\pm ia}}
  \end{equation}
  \begin{proof}
    Heuristically, the result follows from theorem~\ref{thm:a:gauss-integral} by
    replacing $a \to \pm ia$ and using $i = e^{i\frac{\pi}{2}}$.
    The rigorous derivation involves contour integration in the complex plane.
  \end{proof}
\end{thm}

\begin{thm}
  [Shifted Fresnel integral]
  \label{thm:a:shifted-fresnel-integral}
  Let $a, b \in \mathbb{R}$.
  \begin{align}
    \label{eqn:a:shifted-fresnel-integral}
    \intfty e^{-iax^2 + bx} \dx = \sqrt{\frac{\pi}{ia}} e^{i \frac{b^2}{4a}}
  \end{align}
  \begin{proof}
    By completing the square:
    \begin{equation*}
      -iax^2 + bx = -ia\left(x - \frac{b}{2a}\right)^2 + \frac{b^2}{4a}
    \end{equation*}
    By substituting $y = x - \frac{b}{2a}$:
    \begin{align*}
      \intfty e^{-iax^2 + bx} \dx
       & = \intfty \exp\left(-ia\left(x - \frac{b}{2a}\right)^2 + \frac{b^2}{4a}\right) \dx \\
       & = e^{i \frac{b^2}{4a}} \intfty e^{-iay^2} \dy
    \end{align*}
    Theorem~\ref{thm:a:fresnel-integral} gives the result.
  \end{proof}
\end{thm}

\subsection{The harmonic oscillator}

\begin{align*}
  L
   & = \frac{1}{2}m\dot{x}^2 - \frac{1}{2}m\omega^2 x^2                                                        \\
  S\left[ x \right]
   & = \int_{0}^{t} \frac{1}{2}m\dot{x}(\tpr)^2 - \frac{1}{2}m\omega^2 x\tpr \dtpr                             \\
  x(\tpr)
   & = x_{\text{cl}}(\tpr) + \delta x(\tpr),\ \delta x(0) = \delta x(t) = 0                                    \\
  S \left[ x \right]
   & = \int_{0}^{t} \frac{1}{2}m\dot{x}_{\text{cl}}(\tpr)^2 - \frac{1}{2}m\omega^2 x_{\text{cl}}(\tpr)^2 \dtpr
\end{align*}
The linear term vanishes due to the stationarity of the action.
The quadratic term is:
\begin{align*}
  \int_{0}^{t} \frac{1}{2}m\dot{\delta x}(\tpr)^2 - \frac{1}{2}m\omega^2 \delta x(\tpr)^2 \dtpr
\end{align*}

Therefore, specifically for the harmonic oscillator, the action is:
$$
  S\left[x\right] = S\left[x_{\text{cl}}\right] + S\left[\delta x\right]
$$

First, determine $x_{cl}(\tpr)$.
We're working in Lagrangian mechanics, so we can write the Lagrange equations.
\begin{align*}
  \frac{\partial L}{\partial x} = \ddtpr \frac{\partial L}{\partial\dot{x}}
\end{align*}

\end{document}