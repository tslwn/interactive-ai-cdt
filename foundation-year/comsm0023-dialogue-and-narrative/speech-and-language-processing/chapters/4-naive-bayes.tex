\section{Naive Bayes, Text Classification, and Sentiment}

\subsection{Naive Bayes Classifiers}

\begin{itemize}
      \item Probabilistic, generative, and discriminative classifiers:
      \item[] \begin{itemize}
                  \item A probabilistic classifier outputs the probability that an instance belongs to a class.
                  \item Generative classifiers model the probability that a class generated an instance.
                  \item Discriminative classifiers learn the features that best discriminate between classes.
            \end{itemize}
      \item Bayes' theorem, posterior and prior probabilities, and likelihoods:
      \item[] \begin{itemize}
                  \item Naive Bayes is a probabilistic classifier.
                  \item For a document $d$ and classes $c \in C$, it returns the class $\hat{c}$ that has the maximum posterior probability given the document:
                        \begin{equation}
                              \hat{c} = \argmax_{c \in C} P(c \mid d)
                        \end{equation}
                  \item Bayes' theorem states that the posterior probability is:
                        \begin{equation}
                              P(c \mid d) = \frac{P(d \mid c) P(c)}{P(d)}
                        \end{equation}
                  \item $P(d)$ is the prior probability of the document, which is the same for all classes.
                        Therefore, we can return the class:
                        \begin{equation}
                              \hat{c} = \argmax_{c \in C} P(d \mid c) P(c)
                        \end{equation}
            \end{itemize}
      \item Naive Bayes classifiers' assumptions:
      \item[] \begin{itemize}
                  \item A `bag of words' is an unordered set of words.
                        The `bag of words' assumption is that the order of words in a document does not matter.
                  \item Naive Bayes (conditional independence) assumption.
            \end{itemize}
\end{itemize}

\subsection{Training the Naive Bayes Classifier}

\begin{itemize}
      \item Maximum likelihood estimate (frequency counts)
      \item Smoothing, unknown words, and stop words
      \item Example training algorithm
\end{itemize}

\setcounter{subsection}{3}
\subsection{Optimizing for Sentiment Analysis}

\begin{itemize}
      \item For sentiment classification and other tasks, whether a token occurs in a document is more important than how many times it occurs.
            This is called binomial or binary naive Bayes.
      \item A simple way to handle negation is to prepend `not' to tokens that occur between a negation token and the next punctuation mark.
            Negation can be handled more accurately with parsing.
      \item Sentiment lexicons are lists of tokens annotated with positive or negative sentiment.
            When the training set is sparse or not representative of the test set, lexicon features may generalize better.
\end{itemize}

\subsection{Naive Bayes for other text classification tasks}

Sets of tokens and non-linguistic features may be appropriate for other tasks.
E.g., character or byte n-grams are effective for language identification.
Feature selection is the process of selecting the most informative features.

\subsection{Naive Bayes as a Language Model}

Naive Bayes classifiers can use features that depend on multiple tokens or are non-linguistic.
But naive Bayes classifiers that only use single-token features are similar to language models.
Specifically, they are sets of class-specific unigram language models.

\subsection{Evaluation: Precision, Recall, F-measure}

\begin{itemize}
      \item `True' (human-defined) labels are called gold or gold-standard labels.
      \item A confusion matrix or contingency table represents the precision, recall, and accuracy of a classifier.
      \item The F-measure is one way to combine precision and recall into a single measure.
      \item[] \begin{itemize}
                  \item $F_{\beta} = \frac{(\beta^2 + 1) \cdot \text{precision} \cdot \text{recall}}{\beta^2 \cdot \text{precision} + \text{recall}}$
                  \item $\beta < 1$ favours precision and $\beta > 1$ favours recall.
                  \item $\beta = 1$ equally balances precision and recall.
                  \item $F_{\beta = 1}$ or $F_1 = \frac{2 \cdot \text{precision} \cdot \text{recall}}{\text{precision} + \text{recall}}$ is the most common.
                  \item The F-measure is a weighted harmonic mean of precision and recall.
            \end{itemize}
      \item Multi-class classification
      \item[] \begin{itemize}
                  \item In macroaveraging, we compute the performance for each class, then average over classes.
                        A microaverage is dominated by the most frequent class.
                  \item In microaveraging, we compute the performance for all classes.
                        A microaverage better reflects the class distribution, so it is more appropriate when performance on all classes is equally important.
            \end{itemize}
\end{itemize}

\subsection{Test Sets and Cross-validation}

\begin{itemize}
      \item Thus far, we have described a procedure wherein we split the data into training, test, and development sets.
      \item We can instead use all of the data for training and testing by using $k$-fold cross-validation:
      \item[] \begin{itemize}
                  \item Partition the data into $k$ disjoint subsets (folds).
                  \item For each fold, train the classifier on the remaining $k - 1$ folds and test it on the fold.
                  \item Average the test-set error rate over all folds.
            \end{itemize}
      \item But then we can't examine the data to look for features, because we would be looking at the test set.
      \item Therefore, it is common to split the data into training and test sets, then use $k$-fold cross-validation on the training set.
\end{itemize}

\subsection{Statistical Significance Testing}

\begin{itemize}
      \item How do we compare the performance of two systems $A$ and $B$?
            The effect size $\delta(x)$ is the performance difference of $A$ relative to $B$ on a test set $x$.
            We want to know whether $\delta(x)$ is statistically significant.
      \item The null hypothesis $H_0$ is that $\delta(x) \leq 0$.
            We want to reject the null hypothesis to support the hypothesis that $A$ is better than $B$.
      \item Let $X$ be a random variable over all test sets.
            The p-value is the probability $P(\delta(X) \geq \delta(x) \mid H_0)$.
            A result is statistically significant if the p-value is below a threshold.
      \item Parametric tests like t-tests or ANOVAs make assumptions about the distributions of the test statistic.
            In NLP, we typically use non-parametric tests based on sampling:
      \item[] \begin{itemize}
                  \item Approximate randomization
                  \item The bootstrap test
            \end{itemize}
      \item Paired tests compare two sets of aligned observations, e.g., the performance of two systems on a test set.
\end{itemize}

\subsubsection{The Paired Bootstrap Test}

\begin{itemize}
      \item Bootstrapping is repeatedly sampling with replacement from a set of observations.
            Intuitively, we can create many virtual test sets from an observed test set by sampling from it.
      \item The test set is biased by $\delta(x)$ towards A (the expected value of $\delta(X)$ for the bootstrapped test sets is $\delta(x)$).
            \begin{equation}
                  \text{p-value}(x) = \frac{1}{b} \sum_{i = 1}^{b} \mathbb{I}(\delta(x_i) \geq 2 \delta(x))
            \end{equation}
\end{itemize}

\subsection{Avoiding Harms in Classification}

\begin{itemize}
      \item Representational harms affect demographic groups, e.g., by perpetuating negative sterotypes.
      \item Another kind of harm is censorship, which may disproportionately affect minority groups, e.g., in toxicity detection.
      \item A model card documents its data sources, intended use, performance across demographic groups, etc.
\end{itemize}