\section{Sequence Labelling}

\textit{Parts of speech} (POS) include nouns, verbs, pronounds, adverbs,
conjunctions, participles, and articles.
\textit{Part-of-speech (POS) tagging} is the task of labelling each word in a
sequence with a part-of-speech tag.

\textit{Named entities} are referred to with proper names and include people,
locations, and organisations.
\textit{Named-entity recognition (NER)} is the task of labelling each word in a
sequence with a named-entity tag.

\subsection{Word classes}

\subsection{Part-of-speech tagging}

Part-of-speech tagging is a \textit{disambiguation} task: words are ambiguous
because they have multiple possible parts of speech.
The task is to resolve these ambiguities by choosing the proper tags.
Part-of-speech tagging algorithms are highly accurate.

\subsection{Named-entity recognition}

\subsection{Hidden Markov models}

\textit{Hidden Markov models} (HMMs) are based on Markov chains.
They are probabilistic: given a sequence of observations, they compute the
probability distribution over possible sequences of labels and choose the best
one.

\paragraph{Markov chains}

\begin{dfn}
  [Markov assumption]
  \label{dfn:8:markov-assumption}
  Given a sequence of states $U = u_1 \dots u_{i - 1}$, the probability of a
  state $u_i$ depends only on the previous state $u_{i - 1}$:
  \begin{equation}
    P(u_i = s_j \mid u_1 \dots u_{i - 1}) = P(u_i = s_j \mid u_{i - 1})
  \end{equation}
\end{dfn}

\begin{dfn}
  [Markov chain]
  \label{dfn:8:markov-chain}
  A Markov chain is a tuple $(S, \matr{T}, P)$ where:
  \begin{itemize}
    \item $S = \{ s_i \mid i = 1 .. n_s \}$ is a set of $n_s$ states;
    \item $\matr{T} \in \mathbb{R}^{n_s \times n_s}$ is a transition probability
          matrix where $t_{ij}$ is the probability of transitioning from $s_i$
          to $s_j$, i.e.,
          $\sum_{j = 1}^{n_s} t_{ij} = 1 \ \forall\ i \in 1 .. n_s$;
    \item $P_0 = \{ p_i \mid i = 1 .. n_s \}$ is the initial probability
          distribution where $p_i$ is the probability that the first state is
          $s_i$, i.e.,
          $\sum_{i = 1}^{n_s} p_i = 1$.
  \end{itemize}
\end{dfn}

\paragraph{Hidden Markov models}

A HMM represents both observed and \textit{hidden} events, which are not
observed directly (like part-of-speech tags).

\begin{dfn}
  [Hidden Markov model]
  \label{dfn:8:hidden-markov-model}
  A hidden Markov model is a Markov chain (definition~\ref{dfn:8:markov-chain})
  where additionally:
  \begin{itemize}
    \item $W = w_1 \dots w_{n_w}$ is a sequence of $n_w$ observations, each of
          which is taken from a vocabulary $V = \{ v_i \mid i \in 1 .. n_v \}$;
          and
    \item $p_{ij} = P(w_j \mid s_i)$ is an observation likelihood or emission
          probability, i.e., the probability of observing $w_j$ given $s_i$.
  \end{itemize}
\end{dfn}

As well as the Markov assumption (definition \ref{dfn:8:markov-assumption}), a
HMM assumes \textit{output independence}: that the probability of an observation
depends only on the current state, i.e.,
$P(w_i = v_j \mid u_1 u_2 \dots u_i) = P(w_i = v_j \mid u_i)$.

\subsubsection{HMM tagging and decoding}

The task of determining the most likely sequence of hidden states $U$ given a
sequence of observations $W$ is called \textit{decoding}.
For part-of-speech tagging, this task is (applying Bayes' theorem):

\begin{align}
  \hat{u}_1 \hat{u}_2 \dots \hat{u}_{n_u}
   & = \argmax_{u_1 u_2 \dots u_{n_u}}
  P(u_1 u_2 \dots u_{n_u} \mid w_1 w_2 \dots w_{n_w})                                                           \\[3ex]
   & = \argmax_{u_1 u_2 \dots u_{n_u}}
  \frac{P(w_1 w_2 \dots w_{n_w} \mid u_1 u_2 \dots u_{n_u}) P(u_1 u_2 \dots u_{n_u})}{P(w_1 w_2 \dots w_{n_w})} \\[3ex]
   & = \argmax_{u_1 u_2 \dots u_{n_u}}
  P(w_1 w_2 \dots w_{n_w} \mid u_1 u_2 \dots u_{n_u}) P(u_1 u_2 \dots u_{n_u})
  \label{eqn:8:hmm-tagging-1}
\end{align}
HMM tagging makes two more simplifying assumptions:
\begin{itemize}
  \item The probability of a word depends only on its own tag (is independent of
        the other words in the sequence), i.e.,
        \begin{equation}
          \label{eqn:8:hmm-tagging-assumption-1}
          P(w_1 w_2 \dots w_{n_w} \mid u_1 u_2 \dots u_{n_u})
          \approx \prod_{i = 1}^{n_w} P(w_i \mid u_i)
        \end{equation}
  \item The \textit{bigram} assumption: the probability of a tag depends only on
        the previous tag (is independent of the other tags in the sequence),
        i.e.,
        \begin{equation}
          \label{eqn:8:hmm-tagging-assumption-2}
          P(u_1 u_2 \dots u_{n_u})
          \approx \prod_{i = 1}^{n_u} P(u_i \mid u_{i - 1})
        \end{equation}
\end{itemize}
Substituting equations~\ref{eqn:8:hmm-tagging-assumption-1} and
\ref{eqn:8:hmm-tagging-assumption-2} into equation~\ref{eqn:8:hmm-tagging-1}
gives:
\begin{equation}
  \hat{u}_1 \hat{u}_2 \dots \hat{u}_{n_u}
  = \argmax_{u_1 u_2 \dots u_{n_u}}
  \prod_{i = 1}^{n_w} P(w_i \mid u_i) P(u_i \mid u_{i - 1})
\end{equation}
where $P(w_i \mid u_i)$ is the \textit{emission probability} and
$P(u_i \mid u_{i - 1})$ is the \textit{transition probability} from
definitions~\ref{dfn:8:hidden-markov-model} and \ref{dfn:8:markov-chain},
respectively.

\subsubsection{The Viterbi algorithm}

The Viterbi algorithm is a dynamic programming algorithm that computes the most
likely sequence of labels.

A matrix $X \in \mathbb{R}^{n_w \times n_s}$ is initialised with a column for
each observation and a row for each state.
An element $x_{ij}$ is the probability that the HMM is in state $s_j$ after
observing $w_1 \dots w_i$.

The elements $x_{ij}$ are computed by recursively taking the most likely paths
that lead to their observations given the HMM $\lambda$.
\begin{equation}
  x_{ij} = \max_{u_1 \dots u_{i - 1}}
  P(u_1 \dots u_{i - 1},\ w_1 \dots w_{i},\ u_i = s_j \mid \lambda)
\end{equation}
The most likely path to an element $x_{ij}$ is the most likely path to an
element for the \textit{previous} observation $x_{i - 1, k}$ whose state $s_k$
maximises the the probability of transitioning into state $s_j$ \textit{and}
observing $w_i$, i.e.:
\begin{equation}
  x_{ij} = \max_{k \in 1 .. n_s} x_{i - 1, k}\ t_{kj}\ p_{ji}
\end{equation}
where:
\begin{itemize}
  \item $x_{i - 1, k}$ is the probability of $s_k$ after observing $w_1 \dots w_{i - 1}$;
  \item $t_{kj}$ is the probability of transitioning from $s_k$ to $s_j$; and
  \item $p_{ji}$ is the probability of observing $w_i$ given $s_j$.
\end{itemize}

\subsection{Conditional random fields}

\subsection{Evaluation}

Part-of-speech tagging is evaluated by \textit{accuracy}.
Named-entity recognition is evaluated by \textit{recall}, \textit{precision},
and \textit{$F_1$ measure}.
The statistical significance of a difference between the $F_1$ measures of two
named-entity recognition systems is evaluated by the
\textit{paired bootstrap test} or similar.
The segmentation component of named-entity recognition complicates evaluation.