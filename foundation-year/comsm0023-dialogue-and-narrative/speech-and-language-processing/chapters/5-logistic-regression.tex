\section{Logistic Regression}

\paragraph{Generative and discriminative classifiers}

\begin{itemize}
  \item A \textit{generative} model uses a likelihood term $P(d \mid c)$.
        It expresses how to generate the data $d$ from the class $c$.
  \item A \textit{discriminative} model uses a posterior term $P(c \mid d)$.
        It expresses how to discriminate between classes $c$ given the data $d$.
\end{itemize}

\paragraph{Components of a probabilistic machine-learning classifier}

\begin{itemize}
  \item A \textit{feature representation} of the input $\vec{x}$.
  \item A \textit{classification function} $f(\vec{x}) = \hat{y}$, e.g., sigmoid and softmax.
  \item An \textit{objective function}, e.g., the cross-entropy loss function.
  \item An \textit{optimization algorithm}, e.g., stochastic gradient descent.
\end{itemize}

\subsection{The sigmoid function}

Logistic regression learns a vector of \textit{weights} and a \textit{bias} term or intercept:
\begin{equation}
  \label{eqnClassification}
  z = \biggl( \sum_{i = 1}^{n} w_i x_i \biggr) + b \equiv \vec{w} \cdot \vec{x} + b
\end{equation}

The \textit{sigmoid} or logistic function $\sigma : \mathbb{R} \to [0, 1]$ is:
\begin{equation}
  \label{eqnSigmoid}
  \sigma(z) = \frac{1}{1 + e^{-z}}
\end{equation}
It is differentiable and squashes outliers towards 0 or 1.

For binary classification, we require that $P(y = 0) + P(y = 1) = 1$:
\begin{align}
  \label{eqnSigmoidP}
  P(y = 1 \mid \vec{x}) & = \sigma(\vec{w} \cdot \vec{x} + b)                                          \\
  P(y = 0 \mid \vec{x}) & = 1 - \sigma(\vec{w} \cdot \vec{x} + b) = \sigma(-\vec{w} \cdot \vec{x} - b)
\end{align}

\subsection{Classification with logistic regression}

For binary classification, we call 0.5 the \textit{decision boundary} of $P(y \mid \vec{x})$:
\begin{equation}
  \text{decision}(\vec{x}) =
  \begin{cases}
    1 & \text{if } P(y = 1 \mid \vec{x}) > 0.5 \\
    0 & \text{otherwise}
  \end{cases}
\end{equation}

\setcounter{subsubsection}{1}
\subsubsection{Other classification tasks and features}

For some tasks, it helps to design combinations of features or \textit{feature interactions}.
These can be created automatically from \textit{feature templates}.
\textit{Representation learning} tries to learn features automatically.

\paragraph{Scaling input features}

It is common to \textit{standardize} features to have zero mean and unit variance.
This transformation is called the \textit{z-score}.
\begin{equation}
  \label{eqnZScore}
  x_{ij}^{\prime} = \frac{x_{ij} - \mu_i}{\sigma_i} \ \text{where} \
  \mu_i = \frac{1}{N} \sum_{j = 1}^{N} x_{ij} \ \text{and} \
  \sigma_i = \sqrt{\frac{1}{N} \sum_{j = 1}^{N} (x_{ij} - \mu_i)^2}
\end{equation}

Alternatively, we can \textit{normalize} features by $f : \mathbb{R} \to [0, 1]$:
\begin{equation}
  \label{eqnNormalize}
  x_{ij}^{\prime} = \frac{x_{ij} - \min(x_i)}{\max(x_i) - \min(x_i)}
\end{equation}

\subsubsection{Processing many examples at once}

Let $\{ \vec{x}_j \mid j \in 1 .. N \}$ be a set of input feature vectors and $\matr{X} = (x_{ij})$ a matrix where row $i$ is the feature vector $\vec{x}_i$.
Then the output vector is:
\begin{equation}
  \vec{y} = \matr{X} \vec{w} + b \vec{1}
\end{equation}

\subsubsection{Choosing a classifier}

Naive Bayes assumes conditional independence, i.e., treats correlated features as independent.
Logistic regression properly handles correlated features, so it generally works better on large datasets or documents.
However, naive Bayes is easy to implement and fast to train, because it has no optimization step.

\subsection{Multinomial logistic regression}

\textit{Multinomial} ($k > 2$) logistic regression is sometimes called softmax regression.
It is an example of \textit{hard} classification, i.e., it assigns a single class to each instance.
A \textit{one-hot vector} $\vec{y}$ has $y_i = 1$ and $y_j = 0\ \forall\ i \neq j$.

\subsubsection{Softmax}

The \textit{softmax} function is a generalisation of the sigmoid function to $k$ classes:
\begin{equation}
  \label{eqnSoftmax}
  \text{softmax}(z_i) = \frac{e^{z_i}}{\sum_{j = 1}^{k} e^{z_j}} : i \in 1 .. k
\end{equation}

\subsubsection{Applying softmax in logistic regression}

The probability that an input feature vector $\vec{x}$ belongs to class $y_i$ is:
\begin{equation}
  \label{eqnSoftmax1}
  P(y_i = 1 \mid \vec{x})
  = \text{softmax}(\vec{w}_i \cdot \vec{x} + b_i)
  = \frac{e^{\vec{w}_i \cdot \vec{x} + b_i}}{\sum_{j = 1}^{k} e^{\vec{w}_j \cdot \vec{x} + b_j}}
  : i \in 1 .. k
\end{equation}

The vector of output probabilities for the $k$ classes is:
\begin{equation}
  \label{eqnSoftmax2}
  \vec{\hat{y}} = \text{softmax}(\matr{W} \vec{x} + \vec{b})
\end{equation}

\subsubsection{Features in multinomial logistic regression}

In multinomial logistic regression, there is a weight vector and bias for each of the $k$ classes.
The weight vector $\vec{w}$ depends on the input feature vector and the output class, so it is sometimes written $f(x, y)$.

\subsection{Learning in logistic regression}

The learning task to learn the weight vector $\vec{w}$ and bias $b$ that make the vector of output probabilities $\vec{\hat{y}}$ most similar to the true output $\vec{y}$ for the training data.
Typically, the \textit{distance} between them is measured, which is called the \textit{loss} or cost function.
The \textit{cross-entropy} loss is commonly used for logistic regression and neural networks.
The loss function is minimised by an optimisation algorithm.
\textit{Gradient descent}, e.g., stochastic gradient descent, is commonly used.

\subsection{The cross-entropy loss function}

A loss function that prefers that the true outputs are more likely is an example of \textit{conditional maximum likelihood estimation}.
It maximises the log probability of the true outputs given the input feature vectors.

The cross-entropy loss is:
\begin{equation}
  \label{eqnCrossEntropyLoss}
  L_{CE}(\vec{\hat{y}}, \vec{y}) = - \sum_{i = 1}^{k} y_i \log \hat{y}_i
\end{equation}

\paragraph{Derivation for binary regression}

For $k = 2$ classes, the probability that the output label is correct $P(y \mid x)$ is a Bernoulli distribution (it has two discrete outcomes):
\begin{equation}
  \label{eqnDerivation1}
  P(y \mid x) = \hat{y}^y (1 - \hat{y})^{1 - y}
\end{equation}
The logarithm is a monotonic function.
Therefore, instead of maximising the probability, we can maximise the log probability (or minimise the negative log probability):
\begin{equation}
  \label{eqnDerivation2}
  - \log P(y \mid x) = - y \log \hat{y} - (1 - y) \log (1 - \hat{y})
\end{equation}
This is equivalent to \ref{eqnCrossEntropyLoss} for $k = 2$.

\subsection{Gradient descent}

An optimisation algorithm finds the weights $\theta$ that minimise the loss function.
In logistic regression, $\theta = (\vec{w}, b)$.
For $n$ instances, the weights that minimise the average loss are:
\begin{equation}
  \label{eqnWeightsOptimal}
  \vec{\hat{\theta}} = \argmin_{\vec{\theta}} \frac{1}{n} \sum_{i = 1}^{n} L(f(\vec{x}^{(i)}, \vec{\theta}), \vec{y}^{(i)})
\end{equation}

Gradient descent finds the minimum of a function by incrementing its parameters in the opposite direction in parameter space to the direction with the largest gradient.
The increment is weighted by a \textit{learning rate} $\eta$.
In logistic regression, the loss function is convex (has at most one minimum) and gradient descent is guaranteed to find the global minimum.

The equation to update the weights is:
\begin{equation}
  \label{eqnWeightsUpdate}
  \vec{\theta}^\prime = \vec{\theta} - \eta \nabla_{\vec{\theta}} L(f(x, \vec{\theta}), y)
\end{equation}

\subsubsection{The gradient for logistic regression}

For binary logistic regression, the cross-entropy loss function is:
\begin{equation}
  \label{eqnBinaryCrossEntropy}
  L_{\text{CE}}(\hat{y}, y) = - y \log \hat{y} - (1 - y) \log (1 - \hat{y})
  \quad \text{where} \quad
  \hat{y} = \sigma(\vec{w} \cdot \vec{x} + b)
\end{equation}

Its derivative with respect to $w_i$ is:
\begin{align}
  \frac{\partial L_{\text{CE}}}{\partial w_j}
   & = - y \frac{\partial \log \hat{y}}{\partial w_j} - (1 - y) \frac{\partial \log (1 - \hat{y})}{\partial w_j}                         \\
   & = - \frac{y}{\hat{y}} \frac{\partial \hat{y}}{\partial w_j} - \frac{1 - y}{1 - \hat{y}} \frac{\partial (1 - \hat{y})}{\partial w_j} \\
   & = - \biggl( \frac{y}{\hat{y}} - \frac{1 - y}{1 - \hat{y}} \biggr) \frac{\partial \hat{y}}{\partial w_j}
\end{align}
The derivative of the sigmoid function is:
\begin{equation}
  \label{eqnSigmoidDerivative}
  \frac{\partial \sigma(z)}{\partial z} = \sigma(z) (1 - \sigma(z))
\end{equation}
Therefore:
\begin{align}
  \frac{\partial L_{\text{CE}}}{\partial w_j}
   & =  - (y - \hat{y}) \frac{\partial (\vec{w} \cdot \vec{x} + b)}{\partial w_j} \\
   & = (\hat{y} - y) x_j
\end{align}
I.e., the gradient with respect to a weight $w_j$ is the difference between the true $y$ and the estimated $\hat{y}$ multiplied by the input feature $x_j$.

\subsubsection{The stochastic gradient descent algorithm}

The learning rate $\eta$ is a hyperparameter.
It is common to start with a higher learning rate and slowly decrease it.

\setcounter{subsubsection}{3}
\subsubsection{Mini-batch training}